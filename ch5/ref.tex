\documentclass[../main.tex]{subfiles}

\begin{document}

\onehalfspacing
\setcounter{chapter}{4}

\chapter{Conclusions}

\lltoc % Table of contents only when locally compiled

This thesis focuses broadly on the interpretation of choice behavior that seemingly violates Expected Utility Theory (EUT).
Shortly after the introduction of Expected Utility Theory, economists and psychologists began publishing results of experiments that showed choices made by experimental subjects which apparently violate one or more of the axioms of Expected Utility Theory.
That agents often exhibit choice behavior that violates a deterministic interpretation of EUT is now uncontested.
How often choice patterns which violate EUT occur, whether they are more likely to occur for given choice scenarios than others, or whether agents \enquote{systematically}

I begin Chapter 1 by discussing economists' responses to the experimental evidence presented by \textcite{Grether1979}.
These responses vary from developing new theoretical models, typically ones that nest Expected Utility Theory as a special case such as Rank Dependent Utility \parencite{Quiggin1982} and Regret Theory \parencite{Bell1982, Loomes1982}, to critiques of experimental method and scope, such as the necessary precepts for valid inference of experimental data \parencite{Smith1982} and the \enquote{Flat Maximum} critique \parencite{Harrison1989, Harrison1992}, to the reemergence of stochastic models of choice.




\onlyinsubfile{
\newpage
\printbibliography[segment=5, heading=subbibliography]
}

\end{document}
