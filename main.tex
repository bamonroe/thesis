\documentclass[12pt,a4paper]{report}

\usepackage{subfiles}  % For compiling things into one big document from sub-documents

\usepackage[justification=centering]{caption}
\usepackage{lmodern}
\usepackage[T1]{fontenc}
\usepackage{geometry}
\usepackage[nodisplayskipstretch]{setspace}
\setstretch{1.5}
\usepackage[style = american]{csquotes}	% For quoting
\usepackage[hidelinks]{hyperref} %To make references into links
\usepackage{amsmath}	% For equation stuff
\usepackage{amssymb}	% For math symbols
\usepackage{amsthm}		% For theorem environments
\usepackage{relsize}	% To resize math symbols
\usepackage{bm}			% For Bold math Symbols
\usepackage{nicefrac}	% For nice looking fractions
\usepackage{graphicx}	% For Graphics
\usepackage[justification=centering]{subcaption} % For subcaptions to tables/figures
\usepackage[toc,titletoc,title]{appendix}        % For Appendix numbering
\usepackage[raggedright]{titlesec}
\usepackage{afterpage}

% Trying to fix footnot
\raggedbottom
\usepackage[bottom,stable]{footmisc}

%% FOR ABstract only%
%\usepackage{titling}

% Need to specfy path to figures for the graphicx package
\graphicspath{{ch1/figures/}{ch2/figures/}{ch3/figures/}{ch4/figures/}{figures/}}

% for changes
\usepackage[final]{changes}

% Stuff or code
\usepackage{listings}
\usepackage{color}
\lstset{frame=none,
  language=R,
  aboveskip=1mm,
  belowskip=1mm,
  showstringspaces=false,
  columns=flexible,
  basicstyle={\small\ttfamily},
  numbers=none,
  breaklines=true,
  breakatwhitespace=true,
  tabsize=3
}
\newsavebox{\LstBoxR}
\newsavebox{\LstBoxStata}


% Stuff for Tables
\usepackage{tabularx}
\newcolumntype{Y}{>{\centering\arraybackslash}X} % A centered column type that sucks up excess space
\usepackage{caption}
\usepackage{adjustbox}
\usepackage{booktabs}		% For \toprule, \midrule, and \bottomrule
\usepackage{pgfplotstable}	% Generates a table from a CSV
\pgfplotsset{compat=1.12}
\setlength{\extrarowheight}{.5em}
\usepackage{multirow}

\usepackage[stretch=15]{microtype}	% Read about this, tested it, and was sold.
\usepackage[american]{babel}
\usepackage[backend=biber, bibstyle=IEEEtran, uniquename=false, refsegment=chapter, doi=false, isbn=false, url=false, style=authoryear-comp, maxnames=99]{biblatex}	%For bibliography stuff
%\renewcommand*{\revsdnamepunct}{} 
\DefineBibliographyExtras{english}{\let\finalandcomma=\empty}
\renewbibmacro{in:}{}
\addbibresource{/home/bam/thesis/library.bib}	% The path to the bibliography file

% Deal with spacing between equations and text

% Convenience Commands
\newcommand\CE{\ensuremath{\mathit{CE}}}    % Certainty Equivalent
\newcommand\Prob{\ensuremath{\mathit{Pr}}}  % Probability
\newcommand{\E}{\operatorname{E}}           % Variance Operator
\newcommand{\Var}{\operatorname{Var}}       % Variance Operator
\newcommand\OB{\ensuremath{\succ^{\!*}}}    % Certainty Equivalent
\newcommand{\money}[1]{$\$\!\:#1$}          % Money command
\newcommand{\overbar}[1]{\mkern 1.5mu\overline{\mkern-1.5mu#1\mkern-1.5mu}\mkern 1.5mu} % Command to make a shorter overline/longer bar


% Make a break a column title into 2 lines
\newcommand{\cnline}[2][c]{%
	\setlength{\extrarowheight}{0em}
	\begin{tabular}[#1]{@{}c@{}}#2\end{tabular}
	\setlength{\extrarowheight}{5em}
}

% Table of contents only when locally compiled
\newcommand{\lltoc}[1]{
	\pagenumbering{gobble}
	\makeatletter
	\@starttoc{toc}%
	\makeatother
	\break
	\pagenumbering{arabic}
}
\newcommand{\llltoc}[1]{
	\makeatletter
	\@starttoc{toc}%
	\makeatother
}

% A fix for Chapter numbering with above
\let\oldchapter\chapter
\renewcommand{\chapter}[1]{
	\refstepcounter{chapter}% 
	\oldchapter*{{\huge Chapter \thechapter}\\[1em]#1}
}

% Commands that display text either in or out of subfile
\newcommand{\onlyinsubfile}[1]{#1}
\newcommand{\notinsubfile}[1]{}

\title{Stochastic Models in Experimental Economics}
\author{Brian Albert Monroe}

\begin{document}
% Set up the commands for when full document is compiled together
\renewcommand{\onlyinsubfile}[1]{}
\renewcommand{\notinsubfile}[1]{#1}
\renewcommand{\lltoc}[1]{}
\let\chapter\oldchapter

\pagenumbering{arabic}
\maketitle
\doublespacing


\begin{abstract}

Shortly after the introduction of Expected Utility Theory (EUT), economists and psychologists began publishing results that showed choices made by experimental subjects which apparently violate one or more of the EUT axioms.
I discuss economists' responses to this evidence.
These vary from developing new theoretical models, models that nest EUT as a special case, such as Rank Dependent Utility (RDU) and Regret Theory, as well as models that do not nest EUT, such as Cumulative Prospect Theory, to critiques of experimental methods and scope, to the promotion of stochastic models of choice.
I discuss popular stochastic choice models in depth and evaluate their normative coherence.
I find that the \enquote{Random Preferences} stochastic model fails to make normatively coherent statements, in contrast to the \enquote{Random Error} and \enquote{Tremble} models, which do so.
I demonstrate a method to calculate the unconditional likelihood of choice errors for populations of EUT-compliant and RDU-compliant agents, and show how certain characteristics of the population relate to the likelihood of these choice errors and their costliness in terms of forgone welfare.
I find that elements of the stochastic model that are not related to preference relations tend to have a greater influence on unconditional welfare estimates than the preference parameters themselves.
Finally, I conduct a power analysis of the ability of a lottery battery instrument to correctly classify experimental subjects as employing either EUT or RDU, and the effect of this classification on the accuracy of the estimates of welfare surplus for the subjects.
For large ranges of parameter values for these models, I find that the probability of type I and type II errors in the classification process are not trivial, and can be very costly in terms of welfare surplus.
Additionally I show that for a hypothetical population comprising subjects employing EUT or RDU, we can arrive at more accurate welfare surplus estimates on average by assuming that every subject employs the RDU functional, rather than by first trying to differentiate RDU subjects from EUT subjects.

\end{abstract}

\subfile{acknowledgements/acknowledgements}

\tableofcontents

\subfile{ch1/ch1}
\subfile{ch2/ch2}
\subfile{ch3/ch3}
\subfile{ch4/ch4}
\subfile{ch5/ch5}

\printbibliography


\end{document}
