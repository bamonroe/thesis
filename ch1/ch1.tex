\documentclass[../main.tex]{subfiles}

\begin{document}
\onehalfspacing
\setcounter{chapter}{0}
\chapter{Weird Data in Experiments, and Economists' Reactions to Them}

\lltoc

\section{\enquote{Preference Reversals} and the \texorpdfstring{\textcite{Grether1979}}{Grether \& Plott (1979)}  Experiments}

\textcite{Grether1979} describes and tests for explanations of \enquote{preference reversal} phenomenon that had been observed in previous studies by psychologists, in particular \textcite{Lichtenstein1971, Lichtenstein1973} and \textcite{Lindman1971}.
In these experiments subjects were asked to directly state a preference for one of two bets, termed a $P$ bet and a \$ bet,  and then to state a price at which they would be willing to sell each bet.
These stated preferences generate an implied preference for the two bets.
The observed phenomenon was of subjects stating a preference for either the $P$ or the \$ bet in the direct comparison, and then stating a higher selling price for opposite bet.
This type of behavior was deemed a \enquote{preference reversal} because the stated preferences and the implied preferences were inconsistent.
This preference reversal was said to be incompatible with Expected Utility Theory (EUT), which, assuming a deterministic choice process and ignoring indifference for now, requires that a subject state a higher price for the bet which she selected in the direct comparison.
\textcite[623]{Grether1979} set about to condut \enquote{a series of experiments designed to discredit the psychologists' works as applied to economics} and ended up \enquote{as perplexed as the reader who has just been introduced to the problem.} \parencite*[624]{Grether1979} after failing to substantially reduce the observed inconsistencies in their own controlled experiment.

\textcite{Grether1979} identified 13 possible theoretical criticisms and/or explanations of this phenomenon, of which 3 related to economic theory, 6 were psychological in nature, and 4 were artifacts of experimental method.
Of greatest concern to experimental economists should be the explanations involving experimental method and economic theory.
The possible explanations concerning economic theory included misspecified incentives, income effects, and indifference.
The possible explanations involving experimental method included confusion and misunderstanding, low frequency of errors/small sample sizes, unsophisticated subjects, and, my favorite, that the experimenters were psychologists.
Grether and Plott then detail the ways in which each of these possibilities could potentially lead to the observed seemingly theory-inconsistent data, discuss how the previous literature by psychologists inadequately control for these various possibilities, and how their experiment attempts to impose adequate controls.

In identifying these possibilities, \textcite{Grether1979} touch on aspects of conducting economic laboratory experiments which are later codified as necessary precepts for valid controlled experiments by \textcite{Smith1982}.{\footnotemark} 
These precepts will be elaborated on later.

\addtocounter{footnote}{-1}
\stepcounter{footnote}\footnotetext{
	Notably missing from \textcite{Harrison2001} is a restriction on psychologists from conducting experiments.
}

To better understand the nature of the preference reversal phenomenon as described by \textcite{Grether1979}, the subsequent critiques of their method, and potential accommodations of this seemingly inconsistent data, a description of the relevant details of the experiment is needed.
Subjects were students recruited from economics and political science classes, promised a minimum of \money{5} for participation and that the experiment would take no longer than one hour.

The experiment entailed the subjects making 2 types of choices.
The first asked them to state either a strict preference for a $P$ bet or a \$ bet, where the terms are defined below, or stating that they \enquote{Don't care} across 6 pairs of $P$ and \$ bets.
The second asked subjects to state \enquote{the smallest price for which you would sell a ticket to the following bet} \parencite*[630]{Grether1979} for each $P$ bet and \$ bet across the same 6 pairs.
Though each pair of $P$ bets and \$ bets were for different amounts, the relationship between the $P$ bets and \$ bets remained similar across all 6 pairs.
The $P$ bet, named for being a \enquote{probability} bet, is a lottery between winning a small but positive amount of money, $P_1$ with a high probability (never less than 29/36), and incurring a small loss, $P_0$ on the subject's initial endowment with a low probability (never more than 7/36).
The \$ bet, named for being a \enquote{money} bet, is a lottery between winning a large amount of money, $\$_1$, with a low probability (never more than 18/36), and incurring a small loss, $\$_0$, on the subject's initial endowment with a relatively high probability (never less than 18/36).

\textcite{Grether1979} conducted two different experiments eliciting the above choices which  varied slightly in order to test one of the psychological theories for the preference reversal phenomenon.
In the first experiment the subjects were split into two groups in which one group was asked to make a series of hypothetical decisions, for which they would be paid a guaranteed \money{7}, while the other group was initially endowed \money{7}, but told that their final earnings would depend on one of their choices chosen at random and being played out.
Both groups were asked their preferences for pairs 1, 2, and 3, then asked their minimum selling price for all 12 gambles, then asked their preferences for pairs 4, 5, and 6.
The second experiment was also split into two groups but both groups were paid based on their decisions and they were also asked for \enquote{the exact dollar amount such that you are indifferent between the bet and the amount of money} in addition to being asked for a selling price directly.
The first group was asked for a \enquote{selling price} first while the second group was asked for the \enquote{dollar equivalent} amount first.
This additional set of questions were implemented to control for potential strategic behavior that might be associated with the term \enquote{selling}.

All of the groups that were incentivized with real monetary rewards were told that they would only be paid for one of their choices.
This was meant to combat the income effect of accumulating earnings across many questions.
If a selling price question was selected for payment, the experimenters would use the method detailed in G. M. Becker, DeGroot \& Marschak (1964) (BDM).
In this method, a subject is asked to report the lowest price she is willing to accept to give up her right to play a certain lottery.
The experimenter then selects a \enquote{buying} price from a uniform distribution between two feasible price intervals and if the \enquote{selling} price reported by the subject is less than the selected buying price the subject receives the buying price, otherwise the subject plays out the lottery; in this experiment the random distribution was between \money{0.00} and \money{9.99}.
BDM explain how this type of auction mechanism leads to the subject's true selling price being the strongly dominant response, at least in theory.

Despite having conducted this experiment in expectation of refuting the results of psychologists, \textcite{Grether1979} end up confirming these results with their own experiment.
The lack of a substantial reduction in the proportion of subjects displaying the preference reversal phenomenon, particularly in the groups which had monetary incentives attached to their choices, led \textcite{Grether1979} to suggest that certain assumptions economist had held concerning the structure of preferences may not be valid.
\textcite[623]{Grether1979} remarked concerning the preference reversal phenomenon \enquote{The inconsistency is deeper than the mere lack of transitivity or even stochastic transitivity.
It suggests that no optimization principles of any sort lie behind even the simplest of human choices and that the uniformities in human choice behavior which lie behind market behavior may result from principles which are of a completely different sort from those generally accepted.} 

The totals of the different choices for the groups of incentivized subjects are presented in Table \ref{tb:GP1979-res} below.
Experiment 1 in Table \ref{tb:GP1979-res} only includes the group which was paid with real money, while Experiment 2-1 and Experiment 2-2 in Table \ref{tb:GP1979-res} represent experiment 2 groups 1 and 2, respectively,  where the differences between the groups are detailed above.
The mean difference in reported prices for inconsistent choices from experiment 1 is reported in Table \ref{tb:GP1979-mv}, \textcite{Grether1979} did not report these statistics for experiment 2.
In Table \ref{tb:GP1979-mv} the \enquote{Predicted} preference reversal is that of selecting the $P$ bet in the direct comparison and stating a higher selling price for the \$ bet, and the \enquote{Unpredicted} preference reversal is selecting the \$ bet in the direct comparison and stating a higher selling price for the $P$ bet.


\begin{table}
	\caption{\textcite{Grether1979} - Results for Incentivized Experiments}
	\label{tb:GP1979-res}
	\centering
	\begin{tabularx}{\textwidth}{Ycccccc}
		\multicolumn{3}{c}{Experiment 1} \\\hline
			             &         & \multicolumn{3}{c}{Reservation Prices} &              &                \\[0em]\cmidrule{3-5} 
		Bet              & Choices & Consistent & Inconsistent & Equal      & \%Consistent & \%Inconsistent \\[0em]\hline
		P                &      99 &         26 &           69 &    4       &      26.26\% &        69.70\% \\[-.1em]
		\$               &     174 &        145 &           22 &    7       &      83.33\% &        12.64\% \\[-.1em]
		Indifferent      &       3 &            &              &            &              &                \\[-.1em]\hline
		                                                                                   &                \\[-.5em]
		\multicolumn{3}{c}{Experiment 2}                                                                    \\[0em]\hline
		\multicolumn{2}{l}{\underline{Selling Price}}                                                       \\[0em]
		P                & 44     & 8         & 30          & 6    & 18.18\%   & 68.18\%                    \\[-.1em]
		\$               & 72     & 54        & 15          & 3    & 75.00\%   & 20.83\%                    \\[-.1em]
		Indifferent      & 4      &           &             &      &           &                            \\[-.1em]
		\multicolumn{3}{l}{\underline{Equivalent Price}}                                                    \\[0em]
		P                & 44     & 4         & 34          & 6    & 9.09\%    & 68.18\%                    \\[-.1em]
		\$               & 72     & 59        & 11          & 2    & 81.94\%   & 15.28\%                    \\[-.1em]
		Indifferent      &        &           &             &      &           &                            \\[-.1em]\hline
		                                                                       &                            \\[-.5em]
		\multicolumn{3}{c}{Experiment 2-1}                                                                  \\[0em]\hline
		\multicolumn{2}{l}{\underline{Selling Price}}                                                       \\[0em]
		P                & 44     & 16        & 27          & 1    & 36.36\%   & 61.36\%                    \\[-.1em]
		\$               & 64     & 54        & 9           & 1    & 84.38\%   & 14.06\%                    \\[-.1em]
		Indifferent      & 0      &           &             &      &           &                            \\[-.1em]
		\multicolumn{2}{c}{\underline{Equivalent Price}}                                                    \\[0em]
		P                & 44     & 19        & 22          & 3    & 43.18\%   & 50.00\%                    \\[-.1em]
		\$               & 72     & 51        & 10          & 3    & 79.69\%   & 15.63\%                    \\[-.1em]
		Indifferent      & 0      &           &             &      &           &                            \\[-.1em]\bottomrule
		

	\end{tabularx}
\end{table}

\break

\begin{table}
	\caption{ \textcite{Grether1979} - Experiment 1:\\Mean Values of Reversals (in Dollars)}
	\label{tb:GP1979-mv}
	\centering
	\begin{tabularx}{4.5in}{cccccc}
		    & \multicolumn{2}{c}{Predicted} & & \multicolumn{2}{c}{Unpredicted} \\\cmidrule{2-3}\cmidrule{5-6}
		Bet & Incentives & No Incentives    & & Incentives & No Incentives      \\\hline
		1   &       1.71 & 2.49             & &       0.40 & 0.79               \\ 
		2   &       1.45 & 2.65             & &       0.51 & 0.90               \\ 
		3   &       1.48 & 1.29             & &       1.00 & 0.25               \\ 
		4   &       3.31 & 5.59             & &       3.00 & 0.02               \\ 
		5   &       1.52 & 1.79             & &       0.38 & 0.01               \\ 
		6   &       0.92 & 1.18             & &       0.33 & 0.31               \\\bottomrule

	\end{tabularx}
\end{table}


In Table \ref{tb:GP1979-res} two things are apparent.
The first is the degree of inconsistent choices from subjects who chose the $P$ bet in a binary choice option and then reported a higher price for the \$ bet from exactly the same pair in the BDM task.
Of all the groups with incentivized choices, no fewer than 50\% of those who chose the $P$ bet reported a selling price for the $P$ bet that was at least 1 cent below the selling price reported for the \$ best.
Table \ref{tb:GP1979-mv} shows that the average difference between the elicited selling price and the expected value of the lottery in question ranged from 1 cent to \money{5.59}.
The magnitude of these differences will form an important part of the critique of these experiments by \textcite{Harrison1989, Harrison1992} which will be discussed later.

The second implication from Table \ref{tb:GP1979-res} is the asymmetry of the frequency of choice inconsistencies.
While the maximum proportion of inconsistency for subjects selecting a $P$ bet choice in the binary comparison was about 77%, the maximum proportion of inconsistency for selecting a \$ bet was only about 21%.
Also, the  mean value of the reversal is larger for the \enquote{Predicted} reversal for every bet.
Subsequent critiques of these experiments ignore this asymmetry, suggest that it is meaningless due to a failure to satisfy either the saliency or dominance precepts initially proposed by \textcite{Smith1982}, to be discussed later, or suggest that this asymmetry is predicted by alternative theories to EUT, such as the \enquote{Regret Theory} of  Loomes \& Sugden (1982).

\section{Theoretical Critiques of the \texorpdfstring{\textcite{Grether1979}}{Grether \& Plott (1979)} Experiments}

\textcite{Holt1986} offers an explanation of the preference reversal phenomenon which does not require the forfeiture of transitivity.
Instead, \textcite{Holt1986} proposes that should the independence axiom not hold, subjects are not making choices which are separable, but instead are making choices between compound lotteries.
Choices over compounded lotteries may display the preference reversal phenomenon without being a violation of transitivity.

Take for example a three question scenario which represents the \textcite{Grether1979} instrument, where a subject must make a choice between a $P$ bet and a \$ bet, and the subject is asked to reveal her selling price for both the $P$ bet and the \$ bet using the BDM mechanism.
Suppose also that only one of these three choices will be played out for real earnings.
The expected utility of such a scenario would be the following:

\begin{equation}
	\label{eq:GP-EU}
	\frac{1}{3}\E\lbrace u (w+\tilde{x})\rbrace  + \frac{1}{3}\E\lbrace u (w+\tilde{b}(r_\$;X_\$))\rbrace + \frac{1}{3}\E\lbrace u (w+\tilde{b}(r_P;X_P))\rbrace
\end{equation}

where $w$ is the initial wealth of the subject, $\tilde{x}$ is the random payment determined by the chosen lottery, $X_\$$ or $X_P$, and $\tilde{b}$ is the random payment of BDM mechanism given the elicited selling price $r$ for each respective lottery, $X_\$$ or $X_P$.

Given equation (\ref{eq:GP-EU}) and the validity of the independence axiom, the subject should choose the lottery in the binary comparison which she prefers most, and reveal the minimum price she is willing to sell each lottery in order to maximize her utility.
The ranking of the minimum reservation prices should also correspond to the selected of lottery in the binary choice.
However, suppose that the binary choice came after the selling price elicitations, the subject can be said to be making a choice between two compound lotteries:

(2)	  and   

where  is the lottery resulting from the BDM mechanism between the reservation price  and either the  lotteries.
As such, we can rewrite the last part of both these compound lotteries as follows:

(3)	
and (2) can be rewritten as
(4)	

If independence holds, the subject would only choose the left option if, and only if, she preferred , and vise versa.
Should the independence axiom fail, then this relationship also fails and the subject may choose the left option even if she doesn't prefer.
If such  a choice is made, then the price revealed for the chosen lottery using the BDM mechanism may be lower than the price solicited for the alternative lottery because the impact of such a choice has been diluted by the compound lottery. 

\textcite{Holt1986} offers an explanation for the preference reversal phenomenon, but makes no comment about the asymmetry of the phenomenon, nor does he suggest an alternative theory to EUT other than to state that \enquote{...any theory of rational choice in such contexts must be derived from a set of axioms that does not include or imply the independence axiom...} (1986 p.514) \textcite{Holt1986} does state that since the choices are diluted by compounding the opportunity cost of an apparent preference reversal is very low, thus the asymmetry may not be so interesting.
This point is elaborated on by \textcite{Harrison1989, Harrison1992} and will be discussed later. 

\textcite{Karni1987} also state that the preference reversal phenomenon can be explained by an alternative to EUT which doesn't include the independence axiom.
Differing from \textcite{Holt1986} however, they propose a model which would explain the preference reversal phenomenon which they deem\enquote{Expected Utility with Rank Dependent Utilities} (EURDU).
EURDU requires that  the independence axiom not hold, that the possible outcomes of a given lottery be ranked, the probabilities of each of the outcomes be transformed, and weights of the outcomes generated by these transformed probabilities be dependent on the rank of the outcome.
The first axiomatization of a EURDU model was by \textcite{Quiggin1982}, though it was calle \enquote{Anticipated Utility} at the time, and now is more commonly called \enquote{Rank Dependent Utility} (RDU).
\textcite{Yaari1987} independently proposed a utility theory which required a violation of the independence axiom, a ranking of outcomes and a linear utility function called \enquote{Dual Theory}.

Assuming some discrete lottery, , that the probabilities associated with the outcomes of the lotteries sum to 1, and that , then the following function represents the  EURDU of such a lottery, \textcite{Karni1987}:

(1)	

The above utility structure combined with the compound lottery structure, necessitated by the lack of the independence axiom and the BDM mechanism, is sufficient to generate utility maximizing choices which appear to be preference reversals. To demonstrate this, \textcite{Karni1987} proposed a simplified version of the BDM mechanism in which a subject must select a selling price, s, from the set , and the experimenter randomly selects an buying price from the set  to determine the outcome.
Recall that if the buying price is greater than the selling price, the subject receives the buying price, otherwise she plays out the lottery.
In addition, they proposed the following candidate probability weighting function and utility function :

(2)	

(3)	

Using two lotteries from the \textcite{Grether1979} experiments (A and B below), they set up the value function:

(4)	

(5)	

where lottery \enquote{A} is a $P$ bet,  lottery \enquote{B} is a \$ bet, and  returns the selling price of lottery A elicited from the BDM mechanism.

Solving the above equations for the possible values of s produces the following table from \textcite[679]{Karni1987}:
Table 3 - \textcite{Karni1987}


In row 1 of Table 3 the conditional values of A and B are equivalent to the unconditional values  of A and B because if a subject could have chosen a price greater than 5 in this example the subject would have played out either A or B with 100\% probability.
The asterisks indicate at which values of s the conditional value of A or B is maximized.
Row 7 indicates the certainty equivalents A and B which are a direct result of the monotonicity of  and by definition  for every  which is an element of the set of lotteries offered. 

From row 1 we can see that the subject prefers lottery A to B and should choose A in the direct lottery choice.
From row 7 we can see that the true certainty equivalent of lottery A is greater than lottery B.
From rows 3 and 4 we see that when the subject is asked for a selling price for each lottery she would choose a price between 2 and 3 for lottery A, and a price between 3 and 4 for lottery B.
The true certainty equivalent of lottery A is out of the range of selling prices which would maximize this subject's utility for this compound lottery, and the utility maximizing selling price for B is greater than that of A.
This subject would therefore display a \enquote{preference reversal} by selecting A in the direct comparison and selecting a higher price for B with the BDM mechanism.

Using the penny grid employed in the \textcite{Grether1979} experiments, \textcite[680]{Karni1987} calculated the selling price that would have been elicited from the same subject for lottery Aas \money{3.43}, and for lottery B as \money{4.33}.
Because the unconditional values of lottery A and lottery B along with their respective certainty equivalents are the same as in the first row and seventh row of Table 3, these elicited selling prices also display a \enquote{preference reversal}.
The elicited selling prices using this penny grid are also significantly different from the certainty equivalents as calculated in row 7.

In contrast to the critiques of \textcite{Holt1986} and \textcite{Karni1987}, \textcite{Loomes1989} suggest that the apparent preference reversals are in fact violations of transitivity as \textcite[623]{Grether1979} had stated and offer \enquote{Regret Theory} as a potential explanation, which predicts the apparent preference reversal and lacks the transitivity axiom.
Regret Theory was originally and independently developed by \textcite{Loomes1982} and \textcite{Bell1982}, then axiomized by \textcite{Fishburn1987} and modified by \textcite{Loomes1987} to include a \enquote{convexity} axiom to be described below.


Regret Theory assumes that a subject has a \enquote{choiceless} utility function, which is unique up to a linear transformation and assigns a real-valued utility number to every conceivable outcome of an action.
This utility function represents the utility the subject would derive from the outcome of an action if she experienced it without having chosen it \textcite[807]{Loomes1987}.
The\enquote{choiceless} utility of an outcome that would occur in a particular state of the world given an action is compared with the \enquote{choiceless} utility of an outcome that would occur in the same state of the world given another feasible action using a \enquote{modified} utility function:

(1)	

where  is the modified utility function,  and  are the \enquote{choiceless} utilities of the outcome of actions  and , respectively, in the event that state of the world  occurs, and  is the resulting modified utility of having chosen action  instead of action. 

\textcite[809]{Loomes1987} assume that the degree of regret only depends on the difference in  he \enquote{choiceless} utilities which would occur in the same state of the world but given different actions.
Thus (1) can be re-written as follows:

(2)	

where  is a \enquote{regret-rejoice} function. 
A subject would select an action which has the greatest expected modified utility; the sum of the probability weighted modified utilities across all potential states of the world.
Keeping with the notation of \textcite{Loomes1989}, will be rewritten as , where  and  are the outcomes of actions  and , respectively, should state of the world  occur.
Thus  incorporates the transformation of the outcomes into \enquote{choiceless} utilities, and those \enquote{choiceless} utilities into the modified utilities.


(3)	

where  and  are actions  and , respectively.
Two important assumptions are made about : first,  is skew-symmetric, i.e. , and second it is convex, i.e. if , then .
This convexity allows subjects to be \enquote{regret} averse and is the basis for the prediction of the preference reversal phenomenon.

Using this notation, \textcite{Loomes1989} denote the $P$ bets and \$ bets as actions, and modify the choice of a selling price for the BDM mechanism as a binary choice between an action which returns a constant amount of money for certain, C, and either the $P$ bet or the \$ bet.

Table 4 -  \textcite{Loomes1989}

where the columns represent 4 states of the world, , with associated probabilities , and the rows represent the different actions, \$ bet, $P$ bet, and a certainty C, with potential outcomes  and  such that .
If  then the first 2 rows correspond to the \$ bet and $P$ bet from \textcite{Grether1979}.
These actions can be made into 3 pairwise choices, , called a\enquote{triple}.
Applying the formula from (3) to the outcomes and probabilities of Table 4 generates the following preference relations:

(4)	
(5)	
(6)	

In the above pairwise choices, the most common preference reversal phenomenon of the \textcite{Grether1979} experiments is observed if the $P$ bet is chosen over the \$ bet, the \$ bet chosen over the C money certainty, and the C money certainty chosen over the $P$ bet.
The less common preference reversal is observed if the \$ bet is chosen over the $P$ bet, the $P$ bet is chosen over the C money certainty, and the C money certainty is chosen over the \$ bet.
The first of these two preference reversals is predicted by Regret Theory when  and  in equations (4),(5), and (6), while the second one is not.
\textcite[143]{Loomes1989} call the first of these preference reversals the\enquote{predicted} preference reversal, and the second the \enquote{unpredicted} preference reversal.

\textcite{Loomes1989} design three experiments to test whether subjects who display the preference reversal phenomenon may follow Regret Theory instead of EUT by utilizing the special case of  and .
The first two experiments confront subjects with different sets of triples, called the \enquote{choice-only} design, while the third experiment had some subjects face the BDM mechanism to elicit selling prices for each lottery as in \textcite{Grether1979}, called the \enquote{standard} design, and some subjects use the \enquote{choice-only} design.
\textcite[142]{Loomes1989} state that their null hypothesis is that subjects make choices in accordance with EUT but make mistakes randomly such that there should be an equal proportion of subjects who display the \enquote{predicted} preference reversal and subjects who display the \enquote{unpredicted} preference reversal for any given triple.
They state that they will reject this EUT null hypothesis in favor of the alternative of Regret Theory if the frequency of subjects displaying the \enquote{predicted} preference reversal is significantly higher than subjects displaying the \enquote{unpredicted} preference reversal to an extent that can't be attributed to chance.

Experiment 1 had 283 subjects, 120 of which also participated in Experiment 2 which was held a few days later and had some chance of loosing money.
Experiment 3 had 186 subjects.
All subjects we randomly assigned to different subsamples, with each subsample assigned to a unique set of triples.
An equal number of subjects were assigned to the \enquote{choice-only} and \enquote{standard} designs in Experiment 3, and the responses from the group that participated in the \enquote{standard} design were imputed into choices as if they had conducted the \enquote{choice-only} design.
The bets used in the pairwise comparisons are presented in table 5 below:

Table 5 -  \textcite{Loomes1989}

The subjects in Experiment 1 were split into 4 subsamples, A, B, C, and D, each of which made pairwise choices across one triple, , where  varied for each subsample as shown in Table 6 below.
The subjects in Experiment 2 were split into 2 subsamples, E and F, each of which faced 4 triples, with each triple containing the  bet and either the  bet or  bet and unique values of  for each triple.
The subsamples for Experiment 2 differed only by the values of  in their triples.
This is also shown in Table 6 below along with the number of subjects who responded with particular patterns of choice for each triple:


Table 6 - \textcite{Loomes1989}

The subjects in Experiment 3 were split into 6 subsamples,  for the \enquote{standard} design and  for the \enquote{choice-only} design.
Each subsample of the same letter designation faced the same \$ bet and $P$ bet, and after imputing the elicited selling prices from the \enquote{standard} design subsamples into a choice of , all subsamples of Experiment 3 faced the same value of .
The purpose of imputing these choices was to provide a direct comparison of the \enquote{standard} design and the  \enquote{choice-only} design.
The number of subjects who displayed each possible choice pattern for Experiment 3 is shown in Table 7 below.
Some subjects in the \enquote{standard} design reported a selling price for a bet which was greater than the largest possible outcome of the bet, or smaller than the smallest possible outcome of the bet.
The number of subjects who did not display this \enquote{perverse} behavior in the \enquote{standard} design is reported in parenthesis next to the total number of subjects displaying a particular choice pattern in Table 7 below.


Table 7 - \textcite{Loomes1989}


where  and  are the elicited selling prices for the \$ bet and $P$ bet , respectively, and  stands for \enquote{chosen over}.

In every experiment and across every subsample there were statistically significantly more subjects displaying the \enquote{predicted} preference reversal than displaying the \enquote{unpredicted} preference reversal, leading \textcite{Loomes1989} to reject the null hypothesis of EUT in favor of the alternative of Regret Theory for every experiment.

Regret Theory is unique in it's explanation of the preference reversal phenomenon in that, unlike the alternative proposed by \textcite{Karni1987}, Regret Theory predicts both the preference reversal phenomenon's existence as well as the asymmetrical distribution between the two possible patterns of preference reversal. 


\section{Necessary Precepts for Valid Inferences from Economic Experiments, and the Violation of these Precepts}

\textcite{Smith1982} lays out a conceptual framework for modeling microeconomic systems, such as an economic experiment, in terms of an interactive relationship between an environment, an institution, and agent behavior.
In this framework agents send messages to the institution which then maps those messages according to pre-set rules into commodity outcomes.
This framework provides the valuable insight that \enquote{\textelp{} agents do not choose direct commodity allocations.
Agents choose messages, and institutions determine allocations via the rules that carry messages into allocation} [emphasis in the original]\parencite[926]{Smith1982}.
Thus the data which we observe from an experiment is derived from message producing behavior which is said to be a function of the environment for that agent and the institution.
This mapping of messages into allocations does provide the experimenter with valuable information, but only if four sufficient conditions are met for a controlled microeconomic experiment as discussed in \textcite{Smith1982} and \textcite{Harrison1989}.

\subsection{Salience and Potential Violations}

Of the four sufficient conditions proposed by \textcite{Smith1982}, the first, non-satiation, is equivalent to the common requirement of most theories of utility that there exists a reward mechanism such that the subject should not be satiated in it, and should be interpreted in the same way.
The second and third are of primary importance in an experiment attempting to elicit what Smith refers to as \enquote{home-grown} preferences; preferences that are not induced by the experimenter in a laboratory, but instead are the subject's own latent preferences from outside the laboratory.


The second sufficient condition is Saliency.
\textcite[931]{Smith1982} defines this as the condition in which\enquote{Individuals are guaranteed the right to claim a reward which is increasing (decreasing) in the goods (bads) outcomes, xi, of an experiment.}
\textcite[223]{Harrison1994} notes that the above definition combined with the non-satiation requirement leads to a mapping of a message m' to a reward v' instead of m'' that maps into v'', whenever v' > v''.
This has also been interpreted by \textcite{Bruner2011} to mean that the manner in which messages are mapped to rewards by the institution is understood by the subject, even if it is stochastic, otherwise there exists an institutional failure to induce values.
In the syntax of \textcite{Smith1982}⁠⁠:

(1)	  
where  represents the property rights of agent i, and the subscript p on the right side of the equality indicates the perceived property rights of agent i.
M represents the language imposed by the institution, i.e. the set of all messages that can be sent, h represents the process which maps messages to rewards, c represents the processes which maps messages to costs, and g is the governing process which indicates at which point events, including the elicitation of messages, will occur from the beginning of the experiment, t0 , to the end of the experiment, T. 

Should any of the elements of  not be equivalent to their corresponding element in , subjects may believe that they are guaranteed the right to claim a reward which differs from the reward that will be granted by the institution given their message.
Or in the Harrison (1994)⁠ definition, it could lead to a message m'' mapping to v'' when v' > v'' due to the subject believing that m'' maps to v' or that v'' > v'.

An example of this apparent mis-mapping of messages and rewards could be that a simple lack of understanding by the agents about their own property rights leads to a lack of salience.
The degree to which the equivalence of (1) holds could depend on both the complexity of the property right endowment and the ability of the agent to comprehend her own endowment.
Various tax incentives, for instance, endow a proportion of a population with potentially large savings via a tax credit or deduction conditional on citizens behaving in a certain way.
However, these incentives may be too complex to comprehend and thus do not motivate citizens to make the choices that the policy intended even if the citizens would otherwise be willing.

A more complex way in which the equivalence of (1) would fail is a fundamental mistrust of the experimenter.
That is, the subject may fully understand the institution's communication of the  elements of the property right endowment, but doesn't believe that the institution will uphold these rights.
This in a way alludes to one of the experimental theories proposed by \textcite[629]{Grether1979}, that a potential cause of preference reversals in experiments conducted by psychologists was that subjects were in experiments conducted by psychologists: 
\enquote{Subjects nearly always speculate about the purposes of experiments and psychologists have the reputation for deceiving subjects.
It is also well known that subjects' choices are often influenced by what they perceive to be the purpose of the experiment.} 

Economic experiments designed to incentivize subjects to reveal their preferences generally rely on the assumption that the subject views the experimenter to be indifferent to the outcome of the experiment.
\textcite{Schneeweiss1973} explores this view as a alternative explanation of the \textcite{Ellsberg1961} paradox critique of the axioms of \textcite{Savage1954}.
The subjects of the \textcite{Ellsberg1961} experiments, rather than adopting different preferences for events which were \enquote{ambiguous} as opposed to simply uncertain, could view the experiment as a zero-sum two-person game between the experimenter and themselves.
\textcite{Schneeweiss1973} shows that if the subjects assume that the experimenter strategically wants to minimize the expected payout to subjects, the seemingly paradoxical behavior of the subjects can be explained as game theoretic optimal choice behavior.

\textcite{Kadane1992} also notes that if subjects in experiments are skeptical of the intentions of the experimenter, then the seemingly paradoxical behavior described in \textcite{Ellsberg1961} and \textcite{Allais1953} can be explained as a rational response to the possibility of being cheated.
It can be shown that should the agent assign any positive probability to the possibility of the experimenter selecting an outcome that would lead to the lowest expected payout given the subject's choice, both paradoxes fail to violate the axioms of \textcite{Savage1954}.

Though both the above examples require subjects to assume a profit maximizing experimenter, an agent does not need to believe the experimenter has selfish interests for there to be a disconnect between the property rights, as perceived by the agent, and the property rights intended to be induced by the institution.
\textcite[178]{Harrison2006} note that in experiments attempting to describe behavior\enquote{variously labeled 'cooperation,' 'altruism,' 'reciprocity' or 'confusion'}, the experimental methods used are often confounded by the manner in which the experimenter deals with the \enquote{residual} money left on the table after the experiment is over.
If subjects incorporate the residual claimant into their preference structure, subjects displaying \enquote{altruistic} behavior may in fact be attempting to manipulate the residual to the claimant.

In the bulk of these kinds of experiments the experimenter is the implied residual claimant, though not always.
\textcite{Harrison2006} conducted an experiment utilizing the popular \enquote{Dictator} game with four treatments which allowed for variation in both the \enquote{peasant} (the recipient of the money from the Dictator) and in the recipient of the residual funds.
In treatment \enquote{O}, the \enquote{peasant} was another subject paired with the Dictator, and in treatment \enquote{C}, the \enquote{peasant} was an unspecified charity.
In both treatments O and C the residual claimant was implied to be the experimenter (nothing specific was said about the recipient of the residual funds in the instructions to the subjects).
In the \enquote{O(C)} treatment, the \enquote{peasant} was another subject paired with the Dictator with the residual going to an unspecified charity, while in the \enquote{C(O)} treatment the \enquote{peasant} was an unspecified charity with the residual going to another subject randomly selected at the end of the experiment.

\textcite[196]{Harrison2006} find greater giving to the\enquote{peasant} in the C treatment compared to the O treatment, a greater giving to the \enquote{peasant} in the C(O) treatment compared to the C treatment, and a reduction in giving to the \enquote{peasant} in the O(C) treatment compared to the O treatment.
Each of these differences was statistically different from zero.
These results imply that the subjects in this experiment preferred money from the experiment to go to a charity more than to the experimenter, and preferred the money to go to the experimenter more than to the other subjects.
The primary importance of this study is to demonstrate that subjects may incorporate the residual claimant of funds of an experiment into their utility functions. 

Generally, if an agent views any \enquote{third party} (be this the experimenter, a charity, or even Nature) as another agent in the system, while the experimenter views this \enquote{third party} as being outside of the economic system, this could conceivably cause any element in  to differ from its corresponding element in .
Most apparent however is the potential for .

Though the above examples require the agent to view the set of agents in a system differently than the experimenter views the set of agents, this isn't necessary for the agent to believe the institution will not uphold the agent's property right as endowed.
For instance, the agent may believe she has superior knowledge of the mechanism which maps messages to rewards thus causing .
In many experiments, subjects are asked to make a choice between lotteries, with their chosen lottery being played out for a real reward.
The mechanism used to select the outcome of a lottery is almost always some physical device, such as a coin flip, a dice roll, or a bingo cage, the physics of which may be well known by the subject to result in certain outcomes with different likelihoods than the institution suggests; the outcome of flipping a US quarter is not precisely a 50/50 gamble between heads and tails for instance.
In this instance, the choice between lotteries is the message, the outcome of the lottery is the reward, and the mechanism which selects the outcome of the lottery selected is the mechanism which maps the message to the reward.
Asymmetrical beliefs about the reward mechanism can lead to a failure to induce salience as .

Should either the subject or the experimenter not understand the other or should the subject mistrust  the institution, there can be a failure of salience.
However, the latter can be said to be the result of a certain preference structure{\footnotemark}, while the former is usually an experimental artifact.
The institution could fail to properly communicate the endowed property rights, there could be a lack of technical ability on behalf of the subject to comprehend her property rights as endowed, or the actual institutional endowment could be misspecified by the experimenter when observing messages, for instance, by ignoring an agent's perception of additional agents.

\addtocounter{footnote}{-1}
\stepcounter{footnote}\footnotetext{
	For instance, the greater the potential loss to the experimenter (or gain to the agent), the more the agent may prefer to discount fortuitous events and overweight bad events.
	This could be interpreted as mistrust being a determinant of the probability weighting function in Rank Dependent Utility Theory.	
}


Both misunderstanding and mistrust could potentially affect the  messaging behavior of subjects in an experiment.
In the syntax of \textcite{Smith1982}⁠, the degree of (mis)understanding could be reflected by the technology element, , of (2) below, while mistrust would be reflected in the utility element, , of (2) below.
Both elements ultimately shape the behavioral process which determines messages (3):

(2)	
(3)	

In equations (2) and (3),  is called the environment of agent i which is determined by the agents utility structure, , technology endowment, , and wealth endowment, .
An agent's behavior, , then maps the subject's environment conditional on the institutionally granted property right , I , to the message space, .
It is this message space which is observed by the institution and ultimately mapped to a reward.


These two ideas are of course not the only ways in which salience could be experimentally violated.
A much used experimental method is that of asking subjects for responses to hypothetical questions.

There is a general disagreement, even a \enquote{gentle aggression}{\footnotemark}, between experimental economists and experimental psychologists about the use of hypothetical rewards and whether a subject's intrinsic motivation to complete a task proficiently is sufficient to produce a salient mapping of messages to  rewards in an experiment.
\addtocounter{footnote}{-1}
\stepcounter{footnote}\footnotetext{
	The view of \textcite{Hertwig2001} concerning the methodological critique by \textcite{Smith1982} of both experimental economics and experimental psychology.	
}

\textcite[624]{Grether1979} state their case against the use of hypothetical rewards in exploring economic theor:
\enquote{No attempt is made to expand the theory to cover choices from options which yield consequences of no importance.\textelp{} Thus the results of experiments where subjects may be bored, playing games, or otherwise not motivated, present no immediate challenges to theory.} 
This is later discussed specifically in the context of using of imaginary money as a means to provide salience.

\textcite[31]{Camerer1999} note that from 1970-97 not a single experimental study was published in the American Economic Review in which all subjects face only hypothetical rewards, indicating that economists are typically hostile to the idea of intrinsic motivation being sufficient to produce saliency.
\textcite{Camerer1999} show in their analysis of a non-random sample of the experimental literature that increasing financial incentives from zero to positive but low stakes typically improves performance over some domain of experimental tasks, in particular tasks which are effort-responsive like judgment, problem-solving, or clerical tasks, but they find that increasing stakes from some low level to a relatively higher level does little to improve performance and sometimes hinders it.
With respect to tasks for which there is no normative level of performance to be measured, such as games, auctions or choices between risky lotteries as in the \textcite{Grether1979} experiments, \textcite[34]{Camerer1999} state tha\enquote{the most typical result is that incentives do not affect mean performance, but incentives often reduce variance in responses.}
A reduction of variance in responses could lead to a reduction in apparent violations of EUT such as the preference reversals of \textcite{Grether1979}.

\textcite[8]{Camerer1999} state however, that \enquote{The extreme positions, that \textins{material} incentives make no difference at all, or always eliminate persistent irrationalities, are false.}, and in no uncertain terms state \enquote{There is no replicated study in which a theory of rational choice was rejected at low stakes in favor or of a well-specified behavioral alternative, and accepted at high stakes \textelp{} and nothing in any sensible understanding of human psychology suggests that it would [be].} [emphasis in the original], \parencite*[33-34]{Camerer1999}.
This echoes earlier statements by \textcite[246]{Smith1993} that \enquote{\textelp{} rewards matter, and that neither of the polar views - only reward matters, or reward does not matter - are sustainable across the range of experimental economics} and that this view is \enquote{common sense}.

Should a subject have no intrinsic motivation to respond to a task but values money, the introduction of monetary rewards for responses can potentially induce saliency by changing the mapping of messages from non-valued hypothetical rewards to valued monetary rewards.
If a subject does have some degree of intrinsic motivation to respond to a task \enquote{correctly} but also values money, the introduction of monetary rewards for responses can potentially make the gross reward of sending a certain message \enquote{dominate} the subjective costs of sending that message when intrinsic motivation alone wouldn't have sufficed.

\section{Dominance and Potential Violations}

Dominance is the third necessary precept of conducting a valid experiment proposed by \textcite{Smith1982} and elaborated on b \textcite{Harrison1989, Harrison1992}.
\textcite[934]{Smith1982} defines dominance as selecting rewards such th \enquote{The reward structure dominates any subjective costs (or values) associated with participation in the activities of an experiment.}
\textcite[1426]{Harrison1992} refines this definition and states that dominanc \enquote{requires that the reward of sending a message corresponding to a null hypothesis be \enquote{perceptibly and motivationally greater} than the reward of sending an alternative hypothesis.}

If the messages corresponding to the null and alternative hypotheses are  and , respectively, then dominance requires that the values associated with sending each of these messages,  and  respectively, be such that , where  is the subjective cost to the agent of sending message  instead of message \parencite[1427]{Harrison1992}.
This concept is illustrated in Figure 1 below from \textcite[1427]{Harrison1992}:



The potential experiments in Figure 1 have been constructed such that the value to the subject of sending the message corresponding to the null hypothesis is equivalent for both experiments.
However, only Experiment B has a value associated with sending the alternative hypothesis that is less than .
Thus only Experiment B satisfies dominance for this particular set of null and alternative hypotheses.

The difference between salience and dominance in the reward medium is obvious in this example: the subject does in fact value the medium and is not satiated in the medium that is being returned to her for sending messages  and  in both experiments, satisfying salience.
But, only in Experiment B is there a sufficient difference in her valuation of the reward medium from sending  instead of   for her to send the message  instead of .


It is easy to extend this idea to a scenario in which one or more composite alternative hypotheses are being tested against a single point-null hypothesis.
In this case there exists a set of messages \enquote{close} to , , which provide rewards that are not sufficiently different from  to warrant sending .
Similarly there will be a set of messages \enquote{far} enough from , , such that the value of sending any of these messages is sufficiently different from  that the subject would be motivated to send messages from the set of  instead of .
This can be seen in Figure 2 from \textcite{Harrison1992} above.

There is no reason to believe that  shouldn't vary from subject to subject or from task to task.
Looking at Figure 1, Experiment B clearly satisfies dominance for the subject in question and the task requiring the sending of either  or , but a different subject might have a larger subjective cost of sending  instead of  making  and causing a failure of dominance.
Looking at Figure 2, the set of messages in  correspond to those messages which provide rewards which are valued sufficiently differently from  to dominate the subjective cost of sending message , but it cannot be said that any message in  satisfies dominance for any message in .


\textcite{Harrison1989} argues that if message  is the optimal choice for a subject conforming to EUT in a particular experiment, an observance of the choice of  is only relevant as a critique of EUT if dominance is satisfied for that task.
\textcite{Harrison1994} replicated the experiments of\textcite{Grether1979} with only minor differences and observed roughly the same proportion of subjects displaying the apparent preference reversal phenomenon.
However, assuming the subject correctly reported their direct preference for either the $P$ bet of the \$ bet, but mis-reported their true selling price with the BDM mechanism, the difference in expected income for a subject of displaying a preference reversal versus the expected income if they have reported a consistent selling price averaged only \money{0.006}. 
For such a small value of  \enquote{one must nihilistically insist that subjects have a sufficiently low threshold , perhaps even claiming , in order to conclude that such observations allow one to reject the null hypothesis}\parencite[1428]{Harrison1992}.
\textcite[237]{Harrison1994} concludes\enquote{that the subjects in these preference reversal experiments had virtually no incentive to behave any more consistently than they did.}

\section{\texorpdfstring{\textcite{Holt2002}}{Holt \& Laury (2002)} Multiple Price List and Apparent Inconsistencies}

The preference reversal phenomenon of \textcite{Grether1979} is not the only instance of apparent violations of EUT to be replicated by experimentalists.
The \textcite{Ellsberg1961} paradox is a famous early example, as well as repeated over-bidding with respect to the Nash predicted bids in laboratory auctions by \textcite{Cox1982}, \textcite{Cox1983, Cox1983a, Cox1988}.


\textcite[160]{Cox1985} and \textcite[749]{Harrison1989} both note that this overbidding behavior is consistent with risk-averse subjects, but that the bid deviations are too heterogeneous to be consistent with subjects who are uniformly risk-averse.
\textcite{Cox1985} conduct an experiment which attempts to test for heterogeneous risk preferences of subjects and conclude that their experiments provide \enquote{evidence against the compound lottery axiom of EUT} \parencite[165]{Cox1985}.
\textcite{Harrison1989}, though accepting that deviations in bids from Nash predicted outcomes could be caused by risk-averse subjects, argues that the experiments performed did not meet the dominance criteria for a valid experiment.
Thus, the (empirical) question of the degree of heterogeneity in risk preferences among subjects and its influence on bidding behavior.


\textcite[1291]{Hey1994} note that the \enquote{experimentally observed violations of expected utility theory (EUT) have stimulated a deluge of generalized preference functions, almost all containing  EUT as a special case.}
\textcite{Hey1994} conduct a series of experiments on 80 subjects requiring the subjects to state their preference for one lottery across each of 100 lottery pairs to test if subjects conform to EUT (or the \enquote{Dual Theory} of \textcite{Yaari1987}) in favor of risk-neutrality, and whether subjects conform to any of 8 various generalizations of EUT in favor of EUT.
They report substantial evidence against risk-neutrality, and it is clear from their dataset that there is a great deal of heterogeneity in subjects deviating from risk neutrality.
\textcite[1322]{Hey1994} state tha \enquote{we are tempted to conclude by saying that our study indicates that behavior can be reasonably well modeled \textelp{} as \enquote{EU plus noise.}} 

\textcite[1644]{Holt2002} note that bidding behavior in auctions can be used to elicit risk attitudes and that the over-bidding with respect to the Nash predicted outcomes had been attributed to risk aversion.
They propose using a multiple price list (MPL) as a tool for experimentalist to control for individual heterogeneity in risk preferences and conduct an experiment to test whether the extent of risk aversion in subjects is dependent on the stakes of the tasks the subjects are presented.

 
The MPL has been widely used to elicit \enquote{homegrown} preferences for risk for several decades.
The earliest use of the MPL method is considered to be \textcite{Miller1969}, but \textcite{Binswanger1980, Binswanger1981} is regarded as the first experimental economist to identify risk attitudes using an MPL with real payoffs.
The MPL was further developed by \textcite{Schubert1999} and \textcite{Holt2002}.

The instrument employed by HL requires subjects to make a series of binary choices between two lotteries, A and B, across ten lottery pairs.
The instrument used in the \enquote{low-stakes} treatment of HL is as follows:



Table 8 - \textcite{Holt2002}


The logic of the HL MPL is straightforward.
In all ten lottery pairs, the monetary outcomes of  all A lotteries are the same, and similarly with the B lotteries.
Every lottery is comprised of two possible outcomes, with the higher outcome in the B lotteries being greater than the higher outcome in the A lotteries, and the lower outcome in the B lotteries being less than the lower outcome in the A lotteries.
The probability of receiving these outcomes changes from pair to pair.
At the top of the list, the probability of receiving the high amount from each lottery is only 0.1, while the probability of receiving the lower amount is 0.9.
Moving from the top to the bottom of the list, the probability of receiving the higher amount in each lottery increases by 0.1 for each row until at the bottom of the list, row 10, the probability of receiving the higher amount is equal to 1, collapsing the decision to a choice between two certain outcomes.


If a subject has strictly monotonic utility for money, an EUT maximizer should either prefer option A initially and then working down the rows eventually prefer option B for the remaining rows, or the subject should prefer B for all rows.
However, it is often observed that subjects will \enquote{switch} back and forth between selecting lottery A and selecting lottery B.
This is commonly called \enquote{multiple switching behavior} (MSB) \parencite{Bruner2011}.
Data displaying MSB has often been described as \enquote{inconsistent} with an EUT maximization.
\textcite[1645]{Holt2002} describes a subject switching once from A to B when the B lottery becomes sufficiently attractive as what a risk averse subject \enquote{should} do in the HL MPL, implying that MSB is contrary to theory.


\textcite[347]{Harrison2007} notes that a subject could be indifferent between the lotteries of certain pairs, which would explain why a subject displayed MSB.
In fact, under EUT with a deterministic theory of choice specification and some mild assumptions, a subject displaying MSB must be indifferent to the lotteries in lottery pairs between the first switch and the last switch.
This is shown in Appendix A.
\textcite{Harrison2007} conduct an experiment in which some subjects are offered an \enquote{indifferent} option and note that the proportion of subjects who are not offered the indifference option expressing MSB almost equals the proportion of subjects offered the indifference option  selecting the indifference option.
While this is very suggestive that MSB is caused by indifference, it should be noted that the indifference option was played out by the experimenter randomly selecting either option A or option B for payment.
The selection of the indifference option could  therefore represent a preference for a compound lottery of A and B, and not an indifference between A and B.


Choosing option A in row 10 is also an apparent violation of EUT that is not so easily believed to be caused by indifference.
Since there is no uncertainty in the outcomes of either lottery A or lottery B, a choice of A apparently violates the axiom of monotonicity.
It is possible that a subject who chooses A in row 10 has a very good motivation to do so, perhaps because of some influence from outside of the experiment, and could potentially still be making choices in accordance with EUT.
As noted by \textcite[930]{Smith1982}: \enquote{It is hard to find an experimentalist who regards anything as self evident, including the proposition that people prefer more money to less}
This is mentioned merely as a caveat to proclaiming a general loss of rationality on behalf of the subjects making such choices. It is more believable that such subjects are making a mistake in choosing A in row ten.

The extent of this type of behavior in economic experiments is not entirely clear.
Experimental procedures often vary from experimenter to experimenter and form experiment to experiment.
For instance, \textcite{Holt2002} note that some subjects \enquote{crossed out and changed} their responses to choices near their switch point.
If the experimental design hadn't allowed subjects to change their responses, or if the design made it cumbersome to do so, then there might have been more observed \enquote{inconsistent} responses than reported. 

While many experimenters report the number or proportion of subjects switching multiple times, some experimenters exclude subjects who display this behavior from their analysis or only report the number of \enquote{safe} choices (the lottery A choices).
\textcite[1648]{Holt2002} did an analysis without the\enquote{inconsistent} subjects but note that \enquote{The average number of safe choices increases slightly in some treatments when we restrict our attention to those who never switch back, but typically by less than 0.2 choices} and ultimately left \enquote{inconsistent} subjects in their final analysis.
When experimenters only report safe choices, they often don't discriminate between subjects who switched once or multiple times or if one of those \enquote{safe} choices was a choice of A in row 10.
However, if 10 safe choices are reported, then clearly one of them was a choice of lottery A in row 10.

\textcite[9]{Filippin2014} collected datasets of 54 published replications of HL to examine gender differences in estimates for risk attitudes.
\textcite[10-11, 17-18]{Filippin2014} also briefly discuss the number of \enquote{inconsistent} subjects in the aggregated dataset they collected.
Their tables 4 and 7 are combined and reproduced below.

Table 9 - \textcite{Filippin2014}


Table 9 shows to some extent to potential reporting bias of \enquote{inconsistent} behavior in experiments.
Of the 6707 subjects, there was insufficient data to tell if any type of \enquote{inconsistent} behavior occurred with 772 subjects, those with \enquote{Summary} detail, and a further 699 subjects in which it was only possible to tell if there was an apparent violation of monotonicity, those with \enquote{Partial} detail and only reporting the number of safe choices.
About 21.5\% of the data reports insufficient information to determine the extent of inconsistent behavior.

In the first part of Table 9, if we consider only the \enquote{Full} detail data and the \enquote{Partial} detail data which indicates whether or not subjects were \enquote{consistent}, there were 1075 out of 5236 subjects who displayed some sort of inconsistency, about 20.5% of subjects.
\textcite[1647]{Holt2002} reported 28 of 212 (13.2\%) of their subjects exhibited MSB.

While this is not as substantial a proportion of subjects acting in apparent violation of EUT as in the \textcite{Grether1979} preference reversal phenomenon, it is not a trivial proportion.
Additionally most of the potential explanations for the preference reversal phenomenon proposed by \textcite{Grether1979} should be considered resolved given the many replications of HL.
Almost all of the proposed experimental method explanations can be generally rejected: the observance of this behavior is not a low frequency event, nor are sample sizes small; most of the subjects in these experiments (including all of the original HL subjects) were either university students or faculty who can hardly be considered unsophisticated subjects; almost all the experimenters were economists.
The question of confusion or misunderstanding however remains open, along with many of the theoretical critiques.

Two of the theoretical critiques of the \textcite{Grether1979} experiments deserve particular note.
In the HL experiments, subjects were presented with several MPLs and were told that one of their responses would be chosen at random and played out for real earnings.
Most experimenters take advantage of this \enquote{pay one} mechanism in order to increase the stakes for each individual question without breaking their budget.
This \enquote{pay one} mechanism however, requires that an independence axiom hold between choices in for a pattern of choices with a single switch point to be utility maximal across all preferences.
This is no different from the critique noted by \textcite{Holt1986} and \textcite{Karni1987} that should a subject have preferences which are in line with RDU, then there is no apparent violation of economic theory by M.

Similarly, the use of a \enquote{pay one} mechanism may dilute the payoffs of outcomes to the point where the difference in the value of option A versus option B fails to satisfy the dominance criteria of \textcite{Smith1982}.
The HL MPL is structured in such a way that a subject will confront a lottery pair in which she will be near (or entirely) indifferent between the two options.
The \enquote{pay one} mechanism decreases the expected difference in the value of the options by in effect multiplying the probability of each outcome in the lotteries by 0.10. 

The two lottery pairs that the subject is closest to being indifferent to will always be the two on either side of the switch point.
If MSB is party due to a failure to satisfy dominance, it should occur with greater frequency near this point.
\textcite[1648]{Holt2002} \enquote{Even for those who switched back and forth, there is typically a clear division point between clusters of A and B choices, with few 'errors' on each side} and that responses that were crossed out and changed generally were around the switch point \parencite*[1646]{Holt2002}.
\textcite[1647-1648]{Holt2002} further note that the rate of MSB was lowest for their high stakes treatments and highest for their hypothetical stakes treatment.
The way the instrument is built and the frequency of MSB across different treatments suggest that a failure of dominance may be a large factor in explaining the frequency of MSB.

While it may be more believable that dominance is at play for MSB than the non-monotonic choice of lottery A in row 10, dominance failure shouldn't be ruled out a priori for non-monotonic choices.
Take for example, a choice between \money{0.01} and \money{0.02}.
Even if the outcomes of both options are guaranteed, the value difference is likely to be very small, and thus more likely to fail the dominance criteria.
Similarly, because of the \enquote{pay one} mechanism, the expected cost of choosing A in row 10 of the HL MPL is only \money{0.185}.
This may be high in comparison to the cost associated with the generally observed MSB, but it is not unfathomably high.

A failure to induce salience could also help explain some \enquote{inconsistent} choices.
If subjects don't comprehend the two-part payment mechanism (selecting one lottery from the list for payment, then playing the lottery out to determine the reward), or more generally how the probabilities associated with outcomes mapped to (presumably valued) monetary rewards, then there is a failure to induce salience.
A failure to induce salience seems less likely than a failure of dominance when explaining MSB given that most MSB is clustered around what could be called a \enquote{true} switching point and that the remaining choices seem \enquote{consistent} with a salient reward mechanism.
It does however seem more likely that there is a failure of salience for subject who selected lottery A in row 10.

\section{Stochastic Choice as an Explanation of \enquote{Inconsistent} Choices}

It is easy to imagine the theoretical possibility that a subject's preference ordering among a set of alternatives is mapped perfectly without error to the the messages which will realize the optimal outcome for that subject.
In this case, the various theories models have specific predictions as to what elements belong in the chosen set given any set of preferences consistent with that model.
Any occurrence of some alternative within the chosen set which isn't predicted is therefore an apparent violation of the utility theory in question given this mapping assumption.
As has been discussed with respect to the \textcite{Grether1979} and \textcite{Holt2002} replications, the mass of empirical data collected over the past few decades has shown that such apparent violations are common, and that there is a need to \enquote{attempt to modify the theory to account for [these] exception[s] without simultaneously making the theory vacuous} \textcite[634]{Grether1979}.
Such a modification that potentially explains observed choices of subjects which appear to be inconsistent with some predetermined utility theory is that observed choices by subjects are a product of a choice process that is at least in part stochastic, and not wholly deterministic.


The first description of choice as a stochastic process appears to be by \textcite{Edwards1954}, with notable early contributions by \textcite{Luce1958}, \textcite{Debreu1958}, \textcite{Davidson1959}, \textcite{Becker1963}, \textcite{Luce1965}.
Stochastic choice models add elements of randomness to utility models which allow for a degree of error during the evaluation of various alternatives, a degree of randomness of the preference relation used  in the evaluation of alternatives, and/or randomness in the selection of an alternative to belong to the chosen set.
These stochastic models are used as complements to, rather than substitutes for, deterministic theories of utility.
As such, they generally (though not always) seek to link deterministic preference relations, , to probabilities of choice, .

There are various ways to accomplish this linking of ideas, the most common of which are discussed by \textcite{Wilcox2008}.
Generally, these models fall into one of two groups: Random Preference (RP) models, or deterministic preference with a random error models.
The most common deterministic preference with random error models consist of Strong Utility (SU) models, Strict Utility models (a subset of SU models), and Strong Utility's superior derivatives Moderate Utility (MU) models.
I lump SU and MU models into the same group because although they differ on several key points, they both implement stochasticity by assuming some error in the evaluation of alternatives.
They differ by their treatment of this error, with SU models assuming it is homoscedastic and MU models requiring it to be heteroscedastic in a particular fashion over the domain of potential outcomes.
RP models differ from the rest of these by imposing randomness in the preference relation used to evaluate the alternatives.

A notable third group (with only one member I'm aware of) could be considered deterministic preferences with a stochastic choice strategy.
The sole member model in this group was proposed by \textcite{Machina1985} in which randomness is explained as subjects having convex indifference curves in the Machina Triangle space \textcite{Machina1987} and seeking mixtures of alternative lotteries in order to maximize a deterministic preference.
The stochastic mixture is said to be deterministically more preferred to any of the \enquote{pure} lottery options which make up the mixture for such subjects.
This theory implies that there is no error in the evaluation of alternatives, no error in the choice of the stochastic mixture, and no randomness of preferences, while still predicting noisy observed choices.
This theory however has fallen out of favor, and \textcite{Hey1995} provide strong experimental evidence against it.
Because of the large degree of determinant preference behavior in this model, it is not considered in the rest of this text when discussing stochastic choice models.

Another concept called \enquote{trembles}, a term derived from the notion of a \enquote{trembling hand} equilibrium developed by \textcite{Selten1975}, suggests that some choices are made completely at random with no consideration for the underlying values of the alternatives.
Trembles can be implemented by assuming that there is no error in the evaluation of alternatives, no randomness of preferences, and that all of the apparently inconsistent choices are mistakes.
Trembles can also be imposed on top of other stochastic models to transform the \enquote{true} choice probabilities derived from the stochastic models into observed probabilities of choice.
Because in either case the probability of a tremble does not depend on any preference relation, trembles will be left out of the proceeding discussion on stochastic models unless otherwise specifically noted.

In order to discuss stochastic models in more depth, first some notation will be borrowed from \textcite{Wilcox2008}.
Let  and  be two lotteries in lottery pair  which apply discrete probability distributions  and  respectively to a set of  outcomes in .
Let the context of lottery pair , , be the set of outcomes in  with non-zero probabilities applied to them by any lottery in pair .
Assume that for any lottery pair , .
Finally, let  represent the probability that subject  chooses lottery  in pair , and let  equal the probability that subject  chooses lottery  in pair .
It is this concept of probability of choice which is linked to the deterministic concept of the preference relation .


With this notation in place we can define the manner in which the preference relation  is most commonly linked to a probability of choice.
Assuming for now that a subject has an EUT structure: 

		(1)

where  is the utility of prize  given some elements of the vector of structural parameters , and   is a value function determined by properties of lottery  and the vector of structural parameters.
The structural parameters  can be the utilities of the outcomes themselves, or the determining parameters of some parametric function of utility.
Equation (1) can be easily transformed to be represented by rank dependent utility with no loss of generalization.
The  vector would simply have to additionally include parameters defining the probability weights.
Thus, for any transitive structure:

		(2)

RP models posit that for every choice task faced by the subject, a preference relation is drawn randomly from some distribution of preference relations and then the subject makes a choice in accordance with the randomly drawn preference relation.
Econometrically it is possible to model choice such that once the preference relation is drawn, further randomness is added by having the evaluation of the alternatives involve some error process, as in SU and MU models.
However, this is almost never done.
Usually subjects conforming to RP models are said to draw the preference relation randomly from a distribution, and then make a choice deterministically with respect to the preference relation.
Thus the probability of choosing any lottery conditional on this randomly drawn preference is either 0 or 1:

		(3)

where  is the vector of parameters randomly drawn by subject .
Given this relationship, the unconditional choice probability is just the probability of observing  given some joint distribution of the elements of , , where  is a vector of parameters which defines the shape of the distribution.


Let .
Then the unconditional probability that a subject chooses lottery  is:

		(4)

that is, the probability of the choice is simply the probability of observing a  vector which deterministically conforms to that choice.


The collection of SU and MU models represent preferences as stable across choice tasks but with an error of some kind when the utilities of the lotteries are evaluated.
These models are very similar to commonly used latent variable models, with SU models assuming that the latent variable is homoscedastic and MU models assuming that the latent variable is heteroscedastic.
Much in the same way as a standard logit model, SU and MU models state that this latent variable, , relates to the observed choice such that , thus .
The latent variable takes the form:

		(5)

where  is a random variable with a mean of 0, some standard variance and a symmetric c.d.f  where .
Together with , this term represents the degree of noise in the evaluation of alternatives.
Equation (5) is transformed into a choice probability by applying some c.d.f :

		(6)

As  approaches infinity, (6) will approach either 0 or 1, while as  approaches 0, (6) will approach 0.5.
Should the function  take the form of the logistic c.d.f. then the choice probabilities can be calculated by:

		(7)

Because utility structures such as EUT and RDU are unique up to affine transformations,  can be any arbitrarily chosen constant and choice probabilities would still be preference order preserving.
The choice of  is however a defining distinction between SU and MU models.
\textcite{Wilcox2008} proposes and \textcite{Wilcox2011} expands upon a MU model called \enquote{contextual utility} (CU) model. This expands (7) by setting  to the following:

		(8)

where  is the difference in utility of the greatest utility outcome and the least utility outcome in the context of pair , and  can continue to be adjusted with the same effects as  in (6).
It is this property which makes the latent random variable defined in (5) heteroscedastic with respect to the context of the lottery pair.
It has several appealing implications.


First and foremost, it allows the \enquote{more risk averse than} relation derived by \textcite{Pratt1964} for deterministic risky choice to be extended to the\enquote{stochastically more risk averse than} relation across multiple contexts.
\textcite[89]{Wilcox2011} defines it thus: \enquote{Agent a is stochastically more risk averse than agent b \textelp{} iff  for every [mean preserving spread] pair .}
A mean preserving spread (MPS) pair is simply a pair of lotteries with equal expected values.
SU models only allow such relations within contexts, while CU allows for this relation across contexts.
Second, CU only conforms to moderate stochastic transitivity, hence its inclusion as a MU model.
This allows CU to be descriptively more appealing as potential choice patterns which violate strong stochastic transitivity are acceptable with moderate stochastic transitivity.

Stochastic choice models can add a great deal of traction to utility theories which would otherwise falter when applied to apparently inconsistent choice data.
As the difference in value between any two alternatives approaches 0, the choice probabilities of the two alternatives approach each other.
The HL MPL for example is structured in such a way to confront the subject with a list of lottery pairs in which the lotteries get closer and closer in value before diverging in value.
With a deterministic choice process, this structure leads to a choice pattern with a single switch point.
With a stochastic choice process, a choice pattern with a single switch point is only the most likely choice pattern for a subject with preferences in accordance with EUT. 

For any given set of preferences and a modest degree of noise, a choice pattern displaying MSB clustered around the switch point that would be produced by a deterministic choice process is often only marginally less likely than a choice pattern with one switch point.
As noted previously, \textcite[1648]{Holt2002} observed MSB with this kind of clustering.
In this way, stochastic choice models can explain behavior which may otherwise be taken as to imply a failure of dominance.
Also, as noted previously, when the stakes were raised in the HL experiments, the extent of MSB was reduced.
With SU models, this could be potentially explained as the increase in stakes causing an increase in the difference in values of the two alternative lotteries, which would lead to choice probabilities moving closer to 0 or 1.
This however, is not the case with CU models which would normalize the difference to be the same for any scaling of outcomes.


SU and MU models do incorporate an often overlooked idea about what the choices by subjects ultimately amount to.
In particular, it is often suggested, but seldom explored, that occasionally choices by a subject must in fact be inconsistent with the subject's own underlying latent preferences.
As stated by \textcite{Holt1986}⁠:\enquote{Each subject must be making some error or mistake or whatever when answering the questions.}

A choice error can be defined as the selection by a subject of an option among a set of alternatives which does not provide the greatest utility among the set of alternatives.
More generally, to include scenarios where subjects are asked to make multiple selections among alternatives, it is the selection by a subject of an option among a set of alternatives which doesn't belong to the set predicted by a deterministic choice process and some predetermined theory of utility.
As such, errors can only arise conditional on some predetermined theory.


A choice error in this context should not be interpreted as a violation of imposed utility theory.
Any choice error requires that the manner in which the various alternatives are evaluated and ranked be consistent with the subject's latent preference structure, but in the mapping of the subject's preference to the choice space, some noise is introduced that leads to a sub-optimal choice.


Concluding Remarks

Economic orthodoxy over the past half century has been presented with several challenges to the way it characterizes how an agent makes a choice between alternatives in her choice set.
One of these challenges in the form of experiments initially conducted by psychologists, but later replicated by economists, spearheaded the discussion by observing a high frequency \enquote{preference reversals} which seemingly contradicted Expected Utility Theory.
As stated by \textcite{Grether1979}: \enquote{Taken at face value the data are simply inconsistent with preference theory and have broad implications about research priorities in economics.}

The challenge to explain the mounting apparent violations of contemporary economic theory however did not go unheeded.
Theorists noted that it wasn't necessary to forgo the transitivity axiom as suggested by \textcite[623]{Grether1979}, which along with the completeness axiom forms the basis what is considered\enquote{rationality} in economics.
Instead, it was shown by \textcite{Holt1986} and later by \textcite{Karni1987} that should the independence axiom be rejected, there exist preferences which conform to the observed apparent violation of choice.
The \enquote{Anticipated Utility} theory of \textcite{Quiggin1982}, now referred to as \enquote{Rank Dependent Utility}, provides an axiomization of such a theory of utility without the independent axiom.
A more radical departure from Expected Utility Theory was \enquote{Regret  Theory} initially proposed independently by \textcite{Bell1982} and \textcite{Loomes1982}, then axiomized by \textcite{Fishburn1987}, which abandons the transitivity axiom altogether.
\textcite{Loomes1989} test this theory on a replication of the \textcite{Grether1979} experiments and find that it fits the entire dataset better than Expected Utility Theory.

Economics as an empirical science has also progressed greatly over the past half century, notably with \textcite{Smith1982} defining the necessary precepts of conducting a valid controlled economic experiment
Using the framework of these precepts, \textcite{Harrison1994} addresses the \textcite{Grether1979} experiments directly by stating that the dominance criteria put forth by \textcite{Smith1982} wasn't met in the original experiment and that several of the subsequent replications which purport to address the dominance issue do not do so.
Some experimental modifications employed in replications in fact may decrease the saliency of the rewards offered.
In this line of thought, the theory is less to blame than the experiments employed.
\textcite[236-239]{Harrison1994} replicated the\textcite{Grether1979} experiments with some modifications to increase the cost of reporting a selling price inconsistent with a subject's own latent preferences and observed a precipitous drop in the proportion of subjects displaying an apparent preference reversal.


A powerful supplement to current economic theory is the idea of choice as a stochastic processes as opposed to a deterministic one.
Some stochastic models regard apparent violations of economic theory as the result of some randomness of latent preferences while others regard them as \enquote{mistakes} or \enquote{choice errors} from stochastic noise in the evaluation of alternatives.
Such models are powerful as they make only very mild requirements on the structure of underlying preferences if any at all, and create a framework for these preferences to be mapped to choices.
As such most stochastic models are equally applicable to Expected Utility Theory, Rank Dependent Utility, or Prospect Theory \parencite{Kahneman1979, Tversky1992}, and can encompass and enhance the explanatory power of these utility theories.


\section{Appendix A - MSB as Deterministic Indifference}

Lemma (1): If independence holds, and we have  and , then we must also have 


Proof:
Given  and 
			, by independence	
			, by independence	
	, by transitivity of
therefore:


Any three rows of lottery pairs in the \textcite[623]{Grether1979} MPL conform to the following



An example of a multiple switch from a MPL is the following:



A multiple switch in the MPL implies indifference between lotteries of the interior lottery pair:

Since		 
then		, by Lemma(1) and reduction of compound lotteries (ROCL)
therefore	, by completeness of 

If any two lottery pairs in an MPL style instrument display preference relations in the same direction, then every lottery pair which is a linear combination of those two lottery pairs must have the same preference direction if we hold on to EUT and ROCL, otherwise the subject must be indifferent.

\newpage

\printbibliography[segment=1, heading=subbibliography]

\end{document}
