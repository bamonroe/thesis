\documentclass[11pt,a4paper,notitlepage]{report}
\usepackage[left=1in, right=1in, top=1in, bottom=1in]{geometry}

\usepackage{titling}
\usepackage{lipsum}
\usepackage{setspace}	% For line spacing

\pretitle{\begin{center}\Large\bfseries}
\posttitle{\par\end{center}\vskip 0.5em}
%\preauthor{\begin{center}\Large\ttfamily}
\postauthor{\end{center}}
\predate{\par\large\centering}
\postdate{\par}


\title{Stochastic Choice Models and their Relation to Welfare}
\author{Brian Albert Monroe}

\begin{document}

\maketitle

\doublespacing

\begin{abstract}

%Economists attempted to accept the challenge of \textcite[634]{Grether1979} to modify they theory to suitably account for these data while not simultaneously making the theory vacuous.

Shortly after the introduction of Expected Utility Theory, economists and psychologists began publishing results of experiments that showed choices made by experimental subjects which apparently violate one or more of the axioms of Expected Utility Theory.
I begin by discussing economists' responses to this experimental evidence.
These responses vary from developing new theoretical models, typically ones that nest Expected Utility Theory as a special case such as Rank Dependent Utility and Regret Theory, to critiques of experimental method and scope, such as the necessary precepts for valid inference of experimental data and the \enquote{Flat Maximum} critique, to the reemergence of stochastic models of choice.
I proceed to discuss popular stochastic choice models in depth and evaluate their normative coherence.
I find that the \enquote{Random Preferences} stochastic model fails to make normatively coherent statements, while the \enquote{Random Error} and \enquote{Tremble} models do so.
I demonstrate a method to calculate the unconditional likelihood of choice errors for populations of Expected Utility Theory and Rank Dependent Utility agents, and how certain characteristics of the population relate to the likelihood of these choice errors and their costliness in terms of forgone welfare.
I find that elements of the stochastic model that are not related to preference relations tend to have a greater influence on unconditional welfare estimates than the preference parameters themselves.
Finally, I conduct a power analysis of the ability of a lottery battery instrument to correctly classify experimental subjects as employing either Expected Utility Theory or Rank Dependent Utility, and the effect of this classification on the accuracy of the estimates of welfare surplus for the subjects.
For large ranges of parameter values for these models, I find that the probability of type I and type II errors in the classification process are not trivial, and can be very costly in terms of welfare surplus.
Additionally I show that for a hypothetical population comprising subjects employing Expected Utility Theory or Rank Dependent Utility, we can arrive at more accurate welfare surplus estimates in aggregate by assuming every subject employs the Rank Dependent Utility functional rather than first trying to differentiate Rank Dependent Utility subjects from Expected Utility Theory subjects.

\end{abstract}

\end{document}
