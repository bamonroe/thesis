\documentclass{beamer}
\usetheme{Boadilla}

\usepackage{lmodern}
\usepackage{csquotes}
\usepackage{amsmath}	% For equation stuff
\usepackage{amssymb}	% For math symbols
\usepackage{amsthm}		% For theorem environments
\usepackage{relsize}	% To resize math symbols
\usepackage{bm}			% For Bold math Symbols
\usepackage{nicefrac}	% For nice looking fractions
\usepackage{graphicx}	% For Graphics
\usepackage[backend=biber, uniquename=false, doi=false, isbn=false, url=false, style=authoryear-comp, maxnames=99]{biblatex}	%For bibliography stuff
\renewbibmacro{in:}{}
\addbibresource{/home/bam/thesis/library.bib}	% The path to the bibliography file
%For Tables
\usepackage{tabularx}
\newcolumntype{Y}{>{\centering\arraybackslash}X} % A centered column type that sucks up excess space
\usepackage{caption}
\usepackage{adjustbox}
\usepackage{booktabs}		% For \toprule, \midrule, and \bottomrule
\usepackage{pgfplotstable}	% Generates a table from a CSV
\pgfplotsset{compat=1.12}
\setlength{\extrarowheight}{.5em}
\usepackage{multirow}

% Convenience Commands
\newcommand\CE{\ensuremath{\mathit{CE}}}    % Certainty Equivalent
\newcommand\Prob{\ensuremath{\mathit{Pr}}}  % Probability
\newcommand{\E}{\operatorname{E}}           % Variance Operator
\newcommand{\Var}{\operatorname{Var}}       % Variance Operator
\newcommand\OB{\ensuremath{\succ^{\!*}}}    % Certainty Equivalent
\newcommand{\money}[1]{$\$\!\:#1$}          % Money command
\newcommand{\overbar}[1]{\mkern 1.5mu\overline{\mkern-1.5mu#1\mkern-1.5mu}\mkern 1.5mu} % Command to make a shorter over line/longer bar

\title{Welfare Estimation from Economic Experiments}
\author{Brian Albert Monroe}
%\institute{University of Cape Town}
\date{\today}

\begin{document}

\begin{frame}
	\titlepage
\end{frame}

\begin{frame}
\frametitle{Outline}
\tableofcontents
\end{frame}

\begin{frame}
\frametitle{The Grether and Plott Experiments}
\begin{itemize}
	\item \textcite[624]{Grether1979}
	\item \enquote{The inconsistency is deeper than the mere lack of transitivity or even stochastic transitivity. It suggests that no optimization principles of any sort lie behind even the simplest of human choices and that the uniformity in human choice behavior which lie behind market behavior may result from principles which are a completely different sort from those generally accepted}
	\item \textcite[634]{Grether1979}
	\item \enquote{No alternative theory currently available appears to be capable of covering the same extremely broad range of phenomena.}
\end{itemize}
\end{frame}

\begin{frame}
\frametitle{Responses to \textcite{Grether1979}}
\begin{itemize}
	\item \enquote{No alternative theory currently available...}
		\begin{itemize}
			\item Prospect Theory (PT), \textcite{Kahneman1979}
			\item Rank Dependent Utility Theory (RDU), \textcite{Quiggin1982}
			\item Regret Theory, \textcite{Bell1982}; \textcite{Loomes1982}
		\end{itemize}
	\item Experimental Refinement
		\begin{itemize}
			\item \enquote{Necessary precepts for experiments}, \textcite{Smith1982}
			\item \enquote{Flat maximum} critique, \textcite{Harrison1989, Harrison1992}
		\end{itemize}
	\item Stochastic Choice Models
		\begin{itemize}
			\item \enquote{Trembles}, \textcite{Harless1994}
			\item \enquote{Random Errors}, \textcite{Hey1994}
			\item \enquote{Random Preferences}, \textcite{Loomes1995}
		\end{itemize}
\end{itemize}
\end{frame}

\section{Current Practice}

\begin{frame}
\frametitle{Current Practice}
\begin{itemize}
	\item Experiments measuring risk aversion mostly follow the form of \textcite{Hey1994}
		\begin{itemize}
			\item Present subject with a battery of carefully chosen lottery pairs.
			\item Have the subject choose one lottery per pair.
			\item Estimate parameters of several structural utility model given choices.
				\begin{itemize}
					\item Usually some parameterization of EUT or RDU models combined with some stochastic model, often a \enquote{Random Error} derivative.
				\end{itemize}
			\item Select a \enquote{winning} model for each subject.
			\item Calculate welfare surplus? (Not so often)
		\end{itemize}
	\item Lots of research using this general approach
		\begin{itemize}
			\item \textcite{Hey1994, Hey1995, Hey2001}, \textcite{Loomes1995, Loomes1998}, \textcite{Conte2011}, \textcite{Harrison2005}, \textcite{Harrison2005a}, \textcite{Harrison2008}, \textcite{Harrison2016} ...
		\end{itemize}
\end{itemize}
\end{frame}

\begin{frame}
\frametitle{Hey and Orme's Concerns}

\enquote{\textelp{}our study indicates that behavior can be reasonably well modelled (to what might be termed a \enquote{reasonable approximation}) as \enquote{EU plus noise.}}

\hfill \break

\enquote{The inferences that can be drawn \textelp{} about the adequacy or otherwise of EU are not, however, clear cut - mainly because of the large number of generalizations of EU under consideration.
As this research has evolved, and the number of generalizations under consideration has increased, the number of subjects for whom EU emerges as \enquote{the winners} has declined. 
This is inevitable, though it is not clear how one should judge the rate of decline.
\textelp{} Monte Carlo work would be needed to shed more accurate light on such issues}
\end{frame}

\begin{frame}
\frametitle{My Concerns}
\begin{itemize}
	\item Very few power analyses of methods used in model selection from experimental data.
		\begin{itemize}
			\item \textcite{Wilcox2015} an example
		\end{itemize}
	\item No analyses on how costly picking wrong winner is.
		\begin{itemize}
			\item How economists should judge how much this matters
			\item \textcite[25]{Leamer2012}: \enquote{\textins{O}ur goal as economists is not soundness, but usefulness.}
		\end{itemize}
\end{itemize}
\end{frame}

\section{Simulation Approach}

\begin{frame}
\frametitle{Addressing these Concerns - A Simulation Approach}
\begin{itemize}
	\item Replicate a laboratory of human subjects in an artificial laboratory of simulated subjects.
		\begin{itemize}
			\item Simulate economic agent (subject) employing a particular utility function.
			\item Have this subject respond to a lottery pair instrument.
			\item Estimate models on these simulated choices.
			\item Pick a \enquote{Winning} model from among the estimated models.
			\item Repeat 250,000 times per model of interest.
			\item Count how often a subject employing one model classified as another model.
		\end{itemize}
\end{itemize}
\end{frame}

\begin{frame}
\frametitle{Addressing these Concerns - A Simulation Approach}
\begin{itemize}
	\item Need to judge how much this matters
		\begin{itemize}
			\item Have same simulated subjects from before make choices in welfare relevant domain.
			\item Use estimates from winning model model to calculate welfare in this domain.
			\item Calculate the accuracy of these welfare estimate.
		\end{itemize}
\end{itemize}
\end{frame}

\begin{frame}
	\frametitle{Addressing these Concerns - A Simulation Approach}
\begin{itemize}
	\item Picking on \textcite{Harrison2016}
		\begin{itemize}
			\item Critiques \enquote{take-up} metric used in insurance literature to judge policy success.
			\item Argues this metric makes no reference to subjective welfare assessments.
			\item Uses battery of lottery pairs as the estimation domain.
			\item Uses instrument framed as insurance choice decisions as welfare relevant domain.
		\end{itemize}
	\item \textcite{Harrison2016} are utilizing best practices
		\begin{itemize}
			\item 80 lottery pairs
			\item Variation in gradient of indifference curves in Marshak-Machina triangle (MMT)
			\item Lotteries on both interior and boundary of MMT to control for \enquote{boundary effects}.
			\item Real incentives in a range that should matter to subjects (\$10, \$30, \$50)
		\end{itemize}
\end{itemize}
\end{frame}

\begin{frame}
\frametitle{\textcite{Harrison2016} Protocol}
\begin{itemize}
	\item Present subjects lottery battery and insurance task.
	\item Subjects make choices and are paid based on their choices.
	\item Estimate 4 models for each subject, 1 EUT model, 3 RDU models using maximum likelihood.
	\item Test if RDU models are statistically different from EUT.
	\item Of the models statistically different, compare with Log Likelihood of EUT, highest wins.
	\item Use winning model to calculate welfare surplus of each subject's choices in insurance task.
\end{itemize}
\end{frame}

\begin{frame}
\frametitle{\textcite{Harrison2016} models under consideration}
\begin{itemize}
	\item Both EUT and RDU can be expressed as:
	\begin{equation}
		\label{eq4:RDU}
		RDU = \sum_{i=1}^{I} \left[ w_i(p) \times u(x_i) \right]
	\end{equation}
\noindent where $i$ indexes the outcomes, $x_i$, from $\{1,\ldots,I\}$ with $i=1$ being the smallest outcome in the lottery and $i=I$ being the greatest outcome in the lottery, $u(\cdot)$ is a standard utility function, $w_i(\cdot)$ decision weight function applied to outcome $i$ given the distribution of probabilities in the lottery ranked by outcome, $p$.
\end{itemize}
\end{frame}

\begin{frame}
\frametitle{Expected Utility Theory and Rank Dependent Utility Theory}
\begin{itemize}
	\item Constant Relative Risk Aversion utility function:
	\begin{equation}
		\label{eq4:CRRA}
		u(x) = \frac{x^{(1-r)}}{(1-r)}
	\end{equation}
	\item Decision weight function
	\begin{equation}
		\label{eq4:dweight}
		w_i(p) =
		\begin{cases}
			\omega\left(\displaystyle\sum_{j=i}^I p_j\right) - \omega\left(\displaystyle\sum_{k=i+1}^I p_k\right) & \text{for } i<I \\
			\omega(p_i) & \text{for } i = I
		\end{cases}
	\end{equation}
\end{itemize}
\end{frame}

\begin{frame}
\frametitle{Four Models Explored by \textcite{Harrison2016}}
\begin{itemize}
%	\item Many Probability weighing functions available
	\item EUT
		\begin{equation}
			\label{eq4:pw:eut}
			\omega(p_i) = p_i
		\end{equation}
	\item Power \parencite{Quiggin1982}
		\begin{equation}
			\label{eq4:pw:pow}
			\omega(p_i)=p_i^\gamma
		\end{equation}
%		where $\gamma > 0$. 
	\item Inverse S \parencite{Kahneman1979}
		\begin{equation}
			\label{eq4:pw:inv}
			\omega(p_i) = \frac{p_i^\gamma}{\biggl(p_i^\gamma + {(1-p_i)}^\gamma\biggr)^{ \frac{1}{\gamma} } }
		\end{equation}
%		where $\gamma > 0$. 
	\item Flexible 2 parameter function \parencite{Prelec1998}
		\begin{equation}
			\label{eq4:pw:pre}
			\omega(p_i)=\exp(-\beta(-\ln(p_i))^\alpha)
		\end{equation}

	\item $\gamma$, $\alpha$ and $\beta$ all $> 0$.
	\item If the probability weighing parameters equal 1, RDU equivalent to EUT

\end{itemize}
\end{frame}

\begin{frame}
\frametitle{Choice Probabilities}
\begin{equation}
	\label{eq4:RE.f}
	{\Prob}(y_t=j) =\dfrac{\exp\!\left( \dfrac{ G(\beta_n,X_{jt}) }{ D(\beta_n,X_{t})\lambda_n }  \right)}{  \exp\!\left( \dfrac{ G(\beta_n,X_{jt}) }{ D(\beta_n,X_{t})\lambda_n }  \right) + \exp\!\left( \dfrac{ G(\beta_n,X_{kt}) }{ D(\beta_n,X_{t})\lambda_n }  \right)    }
\end{equation}
\begin{itemize}
	\item $D(\cdot)$ is an adjustment function.
	\item $\lambda$ is a \enquote{precision} parameter.
	\item $\lambda \rightarrow \infty$, choice probabilities become equal.
	\item $\lambda \rightarrow 0$, choice probabilities become either 0 or 1.
	
\end{itemize}
\end{frame}

\begin{frame}
\frametitle{Don't Sweat the Math}
\begin{itemize}
	\item CRRA is just a function that can be concave (risk aversion), convex (risk seeking), or linear (risk neutral).
\end{itemize}
\begin{itemize}
	\item RDU
		\begin{itemize}
			\item Conceptually, think pessimistic people overweight the probability of something bad happening.
			\item I make a choice about purchasing car insurance knowing the real probability of an accident, but acting as if that risk is higher.
		\end{itemize}
\end{itemize}
\begin{itemize}
	\item Contextual Utility Stochastic Model
		\begin{itemize}
			\item As the difference in utility grows, the probability of choosing the highest valued option grows.
			\item Some agents generally more attuned to the difference in utilities than others.
		\end{itemize}
\end{itemize}
\end{frame}

\begin{frame}
\frametitle{Simulation Process}
\begin{itemize}
	\item Every lottery pair has 2 alternative lotteries A or B, and every subject employs either EUT or $\mathit{RDU_{Prelec}}$
	\item For each lottery pair, calculate choice probability for A
	\item For each lottery pair, draw a number from uniform distribution over unit interval, [0,1]
	\item If the choice probability of A is greater than the random number, mark the choice as A, otherwise mark it as B
	\item Use maximum likelihood to estimate parameters of the EUT and $\mathit{RDU_{Prelec}}$ models.
		\begin{itemize}
			\item For EUT subjects: $\lbrace r, \lambda \rbrace$
			\item For $\mathit{RDU_{Prelec}}$ subjects: $\lbrace r, \lambda, \alpha, \beta \rbrace$
		\end{itemize}
	\item Pick the winner based on \textcite{Harrison2016} method
\end{itemize}
\end{frame}

\section{Power Results}
\begin{frame}
\frametitle{EUT Subjects}
\begin{figure}[h!]
	\center
	\caption{Probability of \enquote{Winning} for EUT subjects}
	\includegraphics[height = 6.5cm, keepaspectratio]{../figures/HNG_1/win_05-EUT-win-HNG_1.pdf}
	\label{fig:HN1_win_eut}
\end{figure}
\end{frame}

\begin{frame}
\frametitle{ $\mathit{RDU_{Prelec}}$ Subjects}
\begin{figure}[h!]
	\center
	\caption{Probability of \enquote{Winning} for $\mathit{RDU_{Prelec}}$ subjects}
	\includegraphics[height = 6.5cm, keepaspectratio]{../figures/HNG_1/win_05-PRE-win-HNG_1.pdf}
	\label{fig:HN1_win_pre}
\end{figure}
\end{frame}

\section{Welfare Estimation}
\begin{frame}
\frametitle{Welfare Estimation}
\begin{itemize}
	\item Getting to why this matters, welfare surplus.
	\item Certainty Equivalent
\end{itemize}
\begin{align}
	\label{eq4:CEcalc}
	\begin{split}
		&\sum_{i=1}^{I} w_i(p) \frac{x_{ij}^{(1-r)}}{(1-r)} = \frac{ {\CE}_j^{(1-r)}}{(1-r)}\\
		&{\CE}_j =  \left( (1-r) \times \sum_{i=1}^{I} w_i(p) \frac{x_{ij}^{1-r}}{(1-r)} \right)^{ \displaystyle\nicefrac{1}{(1-r)} }
	\end{split}
\end{align}

\end{frame}

\begin{frame}
\frametitle{Welfare Estimation}
\begin{itemize}
	\item Welfare Surplus
	\begin{equation}
		\label{eq4:wsurplus}
		\Delta W_{nt} =  {\CE}_{nyt} - {\CE}_{n1t}^Z ,
	\end{equation}
	\begin{equation}
		\label{eq4:wsurplusT}
		\Delta W_{nT} = \sum_{t=1}^T \left( {\CE}_{nyt} - {\CE}_{n1t}^Z \right)
	\end{equation}
	\item  where ${\CE}_{n1t}^Z$ is the CE of the unchosen option of greatest utility
\end{itemize}
\end{frame}

\begin{frame}
\frametitle{Welfare Estimation}
\begin{itemize}
	\item Difference in Welfare Surplus
	\begin{equation}
		\label{eq4:wsurplusDiff}
		\text{WSD} = \Delta W_{nT}(\hat{\Omega}) - \Delta W_{nT}(\Omega)
	\end{equation}
	\item Used as a measure of welfare estimate accuracy
	\item Also provides a metric of how much we should care about classification
\end{itemize}
\end{frame}
%
%\begin{frame}
%\frametitle{Welfare Estimation}
%\begin{itemize}
%	\item Fit 4 more GAM models
%	\begin{align}
%		\label{eq4:GAM}
%		\begin{split}
%			WSD ~|~ (win = EUT , S = EUT) &= s(r) + s(\lambda)\\
%			WSD ~|~ (win = RDU , S = EUT) &= s(r) + s(\lambda)\\
%			WSD ~|~ (win = EUT , S = RDU) &= s(r) + s(\alpha) + s(\beta) + s(\lambda)\\
%			WSD ~|~ (win = RDU , S = RDU) &= s(r) + s(\alpha) + s(\beta) + s(\lambda)
%		\end{split}
%	\end{align}
%	\item Allows us to predict WSD of winning model, given model employed by subject and parameters
%\end{itemize}
%\end{frame}
%
\begin{frame}
\frametitle{ $\mathit{RDU_{Prelec}}$ Subjects}
\begin{figure}[hb!]
	\center
	\caption{Welfare Surplus Difference of \enquote{Winning} Models for EUT subjects}
	\includegraphics[width=\textwidth]{../figures/HNG_1/win_05-EUT-wel-HNG_1.pdf}
	\label{fig:HN1_wel_eut}
\end{figure}
\end{frame}

\begin{frame}
\frametitle{ $\mathit{RDU_{Prelec}}$ Subjects}
\begin{figure}[hb!]
	\center
	\caption{Welfare Surplus Difference of \enquote{Winning} Models for $\mathit{RDU_{Prelec}}$ subjects}
	\includegraphics[height = 6.5cm, keepaspectratio]{../figures/HNG_1/win_05-PRE-wel-HNG_1.pdf}
	\label{fig:HN1_wel_pre}
\end{figure}
\end{frame}

\section{Alternative Approaches}
\begin{frame}
\frametitle{Alternative Approaches}
\begin{itemize}
	\item Can we do better?
	\item Is problem qualitative, quantitative, or both?
		\begin{itemize}
			\item Better lottery pairs?
			\item More lottery pairs?
			\item Design Change?
			\item Econometrics?
		\end{itemize}
	\item Analysis was repeated on lottery instruments from \textcite{Hey1994}, \textcite{Loomes2002}, \textcite{Schmidt2004}
		\begin{itemize}
			\item 90, 100, 150 lottery pairs, respectively.
			\item No noticeable improvement.
		\end{itemize}
\end{itemize}
\end{frame}

\begin{frame}
\frametitle{Alternative Approaches - Two Proposals}
\begin{itemize}
	\item ${HN}_{C}$ Approach
		\begin{itemize}
			\item Increase the number of choices made by subjects 3, 5, 7, 9, 11 and 13 fold.
				\begin{itemize}
					\item Results in 240, 400, 560, 720, 880, 1040.
				\end{itemize}
			\item Familiar recommendation from statisticians, increase sample size.
		\end{itemize}
	\item \enquote{Default} Approach
		\begin{itemize}
			\item Abandon any effort to classify subjects, focus on maximizing the accuracy of welfare estimates.
			\item Whenever a $\mathit{RDU_{Prelec}}$ model has converged for a subject, use it, otherwise use the EUT model. ($\mathit{RDU_{Prelec}}$ is the \enquote{default} model)
		\end{itemize}
	\item Same simulated subjects from before
\end{itemize}
\end{frame}

\begin{frame}
\frametitle{Alternative Approaches - WSD Expectations}
	\begin{figure}[ht!]
		\center
		\caption{Welfare Surplus Difference for EUT subjects}
		\includegraphics[width=\textwidth]{../figures/EUT-exwel-full.pdf}
		\label{fig:exwel-eut}
	\end{figure}
\end{frame}

\begin{frame}
\frametitle{Alternative Approaches - WSD Expectations}
\begin{figure}[ht!]
	\center
	\caption{Welfare Surplus Difference for $\mathit{RDU_{Prelec}}$ subjects}
	\includegraphics[height = 6.5cm, keepaspectratio]{../figures/PRE-exwel-full.pdf}
	\label{fig:exwel-pre}
\end{figure}
\end{frame}

\begin{frame}
\frametitle{Alternative Approaches}
\begin{itemize}
	\item Any Better? It depends.
		\begin{itemize}
			\item All subjects have more accurate estimates with more lottery pairs.
			\item $\mathit{RDU_{Prelec}}$ Subjects better off under Default approach, EUT subjects slightly worse off.
		\end{itemize}
	\item Choosing between any approach should involve weighing one's priors about the population under investigation, the accuracy of the various potential approaches in producing welfare estimates in a domain that is relevant, and how much tolerance should be given for inaccuracy.
\end{itemize}
\end{frame}

\section{Hypothetical Case}
\begin{frame}
\frametitle{Hypothetical Case}
\begin{itemize}
	\item Consider two populations, one of EUT subjects, one of $\mathit{RDU_{Prelec}}$ subjects
	\item Use pooled estimates from \textcite{Harrison2016} subjects.
		\begin{itemize}
			\item $r       \sim \mathit{N}(0.5, 0.1)$
			\item $\lambda \sim \mathit{Lognormal}(0.1, 0.02)$
			\item $\alpha  \sim \mathit{Lognormal}(1.5, 0.1)$
			\item $\lambda \sim \mathit{Lognormal}(0.7, 0.1)$
		\end{itemize}
	\item Predict average WSD for each of these populations
\end{itemize}
\end{frame}

\begin{frame}
\frametitle{Hypothetical Case}
\begin{table}[h!] 
	\centering
	\caption{Expected Welfare Surplus Difference (WSD), HN Approach}
	\label{tb:HN1_win05_pop}
	\adjustbox{max width = \textwidth}{
	\pgfplotstabletypeset[
		col sep=comma,
		every head row/.style={
			after row=\hline
		},
		display columns/0/.style={
			string type,
			column name = {}
		},
		display columns/1/.style={
			precision = 2,
			fixed,
			zerofill,
			column name = {p(EUT)}
		},
		display columns/2/.style={
			precision = 2,
			fixed,
			zerofill,
			column name = {p(Prelec)}
		},
		display columns/3/.style={
			precision = 2,
			fixed,
			zerofill,
			column name = {$\text{WSD}_{\text{EUT}}$}
		},
		display columns/4/.style={
			precision = 2,
			fixed,
			zerofill,
			column name = {$\text{WSD}_{\text{Prelec}}$}
		},
		display columns/5/.style={
			precision = 2,
			fixed,
			zerofill,
			column name = {Expected WSD}
		},
		]{../tables/HNG_1-win_05_full-table.csv} % path/to/file
	}
\end{table}

\begin{table}[h!] 
	\centering
	\caption{Expected Welfare Surplus Difference (WSD), Default Approach}
	\label{tb:HN1_default_pop}
	\adjustbox{max width = \textwidth}{
	\pgfplotstabletypeset[
		col sep=comma,
		every head row/.style={
			after row=\hline
		},
		display columns/0/.style={
			string type,
			column name = {}
		},
		display columns/1/.style={
			precision = 2,
			fixed,
			zerofill,
			column name = {p(EUT)}
		},
		display columns/2/.style={
			precision = 2,
			fixed,
			zerofill,
			column name = {p(Prelec)}
		},
		display columns/3/.style={
			precision = 2,
			fixed,
			zerofill,
			column name = {$\text{WSD}_{\text{EUT}}$}
		},
		display columns/4/.style={
			precision = 2,
			fixed,
			zerofill,
			column name = {$\text{WSD}_{\text{Prelec}}$}
		},
		display columns/5/.style={
			precision = 2,
			fixed,
			zerofill,
			column name = {Expected WSD}
		},
		]{../tables/HNG_1-default_full-table.csv} % path/to/file
	}
\end{table}
\end{frame}

\begin{frame}
\frametitle{Hypothetical Case}
\begin{table}[h!] 
	\centering
	\caption{Expected Welfare Surplus Difference (WSD), $\text{HN}_{240}$ Approach}
	\label{tb:HN1_win05_pop}
	\adjustbox{max width = \textwidth}{
	\pgfplotstabletypeset[
		col sep=comma,
		every head row/.style={
			after row=\hline
		},
		display columns/0/.style={
			string type,
			column name = {}
		},
		display columns/1/.style={
			precision = 2,
			fixed,
			zerofill,
			column name = {p(EUT)}
		},
		display columns/2/.style={
			precision = 2,
			fixed,
			zerofill,
			column name = {p(Prelec)}
		},
		display columns/3/.style={
			precision = 2,
			fixed,
			zerofill,
			column name = {$\text{WSD}_{\text{EUT}}$}
		},
		display columns/4/.style={
			precision = 2,
			fixed,
			zerofill,
			column name = {$\text{WSD}_{\text{Prelec}}$}
		},
		display columns/5/.style={
			precision = 2,
			fixed,
			zerofill,
			column name = {Expected WSD}
		},
		]{../tables/HNG_3-win_05_full-table.csv} % path/to/file
	}
\end{table}

\begin{table}[h!] 
	\centering
	\caption{Expected Welfare Surplus Difference (WSD), $\text{HN}_{400}$  Approach}
	\label{tb:HN1_default_pop}
	\adjustbox{max width = \textwidth}{
	\pgfplotstabletypeset[
		col sep=comma,
		every head row/.style={
			after row=\hline
		},
		display columns/0/.style={
			string type,
			column name = {}
		},
		display columns/1/.style={
			precision = 2,
			fixed,
			zerofill,
			column name = {p(EUT)}
		},
		display columns/2/.style={
			precision = 2,
			fixed,
			zerofill,
			column name = {p(Prelec)}
		},
		display columns/3/.style={
			precision = 2,
			fixed,
			zerofill,
			column name = {$\text{WSD}_{\text{EUT}}$}
		},
		display columns/4/.style={
			precision = 2,
			fixed,
			zerofill,
			column name = {$\text{WSD}_{\text{Prelec}}$}
		},
		display columns/5/.style={
			precision = 2,
			fixed,
			zerofill,
			column name = {Expected WSD}
		},
		]{../tables/HNG_5-win_05_full-table.csv} % path/to/file
	}
\end{table}
\end{frame}

\begin{frame}
\frametitle{Hypothetical Case}
\begin{itemize}
	\item From  HN to Default
		\begin{itemize}
			\item EUT accuracy decreases by \$0.62.
			\item $\mathit{RDU_{Prelec}}$ increases by \$2.60.
		\end{itemize}
	\item This population would need to be greater than 80.7\% EUT subjects for HN approach to result in more accurate welfare estimates on average.
		\begin{itemize}
			\item Current evidence and power calculations done here don't support that level of EUT subjects.
		\end{itemize}
\end{itemize}
\end{frame}

\section{Caution and Wild Speculation}
\begin{frame}
\frametitle{Caution and Wild Speculation}
\begin{itemize}
	\item \textcite{Gelman2013}: \enquote{Criticism is easy, doing research is hard.}
	\item ${HN}_{1040}$ approach not feasible, ${HN}_{400}$ may not be.
		\begin{itemize}
			\item Experimental environments are limited, and have good reason to remain so.
		\end{itemize}
	\item Subjects almost certainly do not employ these specific utility models
		\begin{itemize}
			\item Not the same thing as saying \enquote{agents do not adhere to EUT or RDU.}
			\item Economic models are ment to be useful approximations of choice behavior in incentivized environments
			\item It could be the case that subjects employ other weird models, and the Default approach leaves them worse off than the HN approach, with both approaches equally \enquote{wrong}.
		\end{itemize}
\end{itemize}
\end{frame}

\begin{frame}
\frametitle{Caution and Wild Speculation}
\begin{itemize}
	\item Assumes we care equally about EUT and $\mathit{RDU_{Prelec}}$ subjects from a policy perspective.
		\begin{itemize}
			\item Subgroups correlated with probability weighting?
		\end{itemize}
	\item EUT is flexible, so where are the power calculations on semi-parametric EUT models?
		\begin{itemize}
			\item Can RDU actually be misidentified \enquote{weird} EUT?
			\item Prospect Theory?
		\end{itemize}
	\item Time preferences, belief calibrations, ...
		\begin{itemize}
			\item Time preferences are estimated jointly with risk preferences
			\item Can misidentification in the one lead to misidentification in the other?
			\item Is hyperbolic discounting misidentified exponential discounting?
			\item Other experimental evidence from jointly estimating risk preferences?
		\end{itemize}
\end{itemize}
\end{frame}

\begin{frame}[allowframebreaks]
\frametitle{References}
\printbibliography[heading=subbibliography]
\end{frame}

\end{document}
