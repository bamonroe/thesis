\documentclass{beamer}
\usetheme{Boadilla}

\usepackage{lmodern}
\usepackage{csquotes}
\usepackage{amsmath}	% For equation stuff
\usepackage{amssymb}	% For math symbols
\usepackage{amsthm}		% For theorem environments
\usepackage{relsize}	% To resize math symbols
\usepackage{bm}			% For Bold math Symbols
\usepackage{nicefrac}	% For nice looking fractions
\usepackage{graphicx}	% For Graphics
\usepackage[backend=biber, uniquename=false, doi=false, isbn=false, url=false, style=authoryear-comp, maxnames=99]{biblatex}	%For bibliography stuff
\renewbibmacro{in:}{}
\addbibresource{/home/bam/thesis/library.bib}	% The path to the bibliography file

\title{Welfare Estimation from Economic Experiments}
\author{Brian Albert Monroe}
%\institute{University of Cape Town}
\date{\today}

\begin{document}

\begin{frame}
	\titlepage
\end{frame}

\begin{frame}
\frametitle{Outline}
\tableofcontents
\end{frame}

\section{Early Evidence of Anomalies}

\begin{frame}
\frametitle{The Grether and Plott Experiments}
\begin{itemize}
	\item \textcite[624]{Grether1979}
	\item \enquote{The inconsistency is deeper than the mere lack of transitivity or even stochastic transitivity. It suggests that no optimization principles of any sort lie behind even the simplest of human choices and that the uniformity in human choice behavior which lie behind market behavior may result from principles which are a completely different sort from those generally accepted}
\end{itemize}
\end{frame}

\begin{frame}
\frametitle{The Grether and Plott Experiments}
\begin{itemize}
	\item \textcite[634]{Grether1979}
	\item \enquote{The fact that preference theory and related theories of optimization are subject to exception does not mean that they should be discarded. No alternative theory currently available appears to be capable of covering the same extremely broad range of phenomena. In a sense the exception is an important discovery, as it stands as an answer to those who would chard that preference theory is circular and/or without empirical content. It also stands as a challenge to theorists who may attempt to modify the theory to account for this exception without simultaneously making the theory vacuous.}
\end{itemize}
\end{frame}

\begin{frame}
\frametitle{Responses to \textcite{Grether1979}}
\begin{itemize}
	\item \enquote{No alternative theory currently available...}
		\begin{itemize}
			\item Prospect Theory (PT), \textcite{Kahneman1979}
			\item Rank Dependent Utility Theory (RDU), \textcite{Quiggin1982}
			\item Regret Theory, \textcite{Bell1982}; \textcite{Loomes1982}
		\end{itemize}
	\item Experimental Refinement
		\begin{itemize}
			\item \enquote{Necessary precepts for experiments}, \textcite{Smith1982}
			\item \enquote{Flat maximum} critique, \textcite{Harrison1989, Harrison1992}
		\end{itemize}
	\item Stochastic Choice Models
		\begin{itemize}
			\item \enquote{Trembles}, \textcite{Harless1994}
			\item \enquote{Random Errors}, \textcite{Hey1994}
			\item \enquote{Random Preferences}, \textcite{Loomes1995}
		\end{itemize}
\end{itemize}
\end{frame}

\section{Current Practice}

\begin{frame}
\frametitle{Current Practice}
\begin{itemize}
	\item Experiments measuring risk aversion mostly follow the form of \textcite{Hey1994}
		\begin{itemize}
			\item Present subject with a battery of carefully chosen lottery pairs.
			\item Have the subject choose one lottery per pair.
			\item Estimate parameters of several structural utility model given choices.
				\begin{itemize}
					\item Usually some parameterization of EUT or RDU models combined with stochastic model, often a \enquote{Random Error} derivative.
				\end{itemize}
			\item Select a \enquote{winning} model for each subject.
			\item Calculate welfare? (Not so often)
		\end{itemize}
	\item Lots of research using this basic approach
		\begin{itemize}
			\item \textcite{Hey1994, Hey1995, Hey2001}, \textcite{Loomes1995, Loomes1998}, \textcite{Conte2011}, \textcite{Harrison2005}, \textcite{Harrison2005a}, \textcite{Harrison2008}, \textcite{Harrison2016} 
		\end{itemize}
\end{itemize}
\end{frame}

\begin{frame}
\frametitle{Hey and Orme's Concerns}
\enquote{The inferences that can be drawn \textelp{} about the adequacy or otherwise of EU are not, however, clear cut - mainly because of the large number of generalizations of EU under consideration.
As this research has evolved, and then number of generalizations under consideration has increased, the number of subjects for whom EU emerges as \enquote{the winners} has declined. 
This is inevitable, though it is not clear how one should judge the rate of decline.
\textelp{} Monte Carlo work would be needed to shed more accurate light on such issues}
\end{frame}

\begin{frame}
\frametitle{My Concerns}
\begin{itemize}
	\item Very few power analyses of methods used in model selection from experimental data.
		\begin{itemize}
			\item \textcite{Wilcox2015} an example
		\end{itemize}
	\item No analyses on how costly picking wrong winner is.
		\begin{itemize}
			\item A lot of work done to aid in accuracy of model selection, is it wasted?
			\item \textcite[25]{Leamer2012}: \enquote{Fortunately, our goal as economists is not soundness, but usefulness.}
		\end{itemize}
\end{itemize}
\end{frame}

\begin{frame}
\frametitle{Expected Utility Theory and Rank Dependent Utility Theory}
\begin{itemize}
	\item Both models can be expressed as:
	\begin{equation}
		\label{eq4:RDU}
		RDU = \sum_{i=1}^{I} \left[ w_i(p) \times u(x_i) \right]
	\end{equation}
\noindent where $i$ indexes the outcomes, $x_i$, from $\{1,\ldots,I\}$ with $i=1$ being the smallest outcome in the lottery and $i=I$ being the greatest outcome in the lottery, $u(\cdot)$ is a standard utility function, $w_i(\cdot)$ decision weight function applied to outcome $i$ given the distribution of probabilities in the lottery ranked by outcome, $p$.
\end{itemize}
\end{frame}

\begin{frame}
\frametitle{Expected Utility Theory and Rank Dependent Utility Theory}
\begin{itemize}
	\item Constant Relative Risk Aversion utility function:
	\begin{equation}
		\label{eq4:CRRA}
		u(x) = \frac{x^{(1-r)}}{(1-r)}
	\end{equation}
	\item Decision weight function
	\begin{equation}
		\label{eq4:dweight}
		w_i(p) =
		\begin{cases}
			\omega\left(\displaystyle\sum_{j=i}^I p_j\right) - \omega\left(\displaystyle\sum_{k=i+1}^I p_k\right) & \text{for } i<I \\
			\omega(p_i) & \text{for } i = I
		\end{cases}
	\end{equation}
\end{itemize}
\end{frame}

\begin{frame}
\frametitle{Four Models Explored by \textcite{Harrison2016}}
\begin{itemize}
%	\item Many Probability weighing functions available
	\item EUT
		\begin{equation}
			\label{eq4:pw:eut}
			\omega(p_i) = p_i
		\end{equation}
	\item Power \parencite{Quiggin1982}
		\begin{equation}
			\label{eq4:pw:pow}
			\omega(p_i)=p_i^\gamma
		\end{equation}
%		where $\gamma > 0$. 
	\item Inverse S \parencite{Kahneman1979}
		\begin{equation}
			\label{eq4:pw:inv}
			\omega(p_i) = \frac{p_i^\gamma}{\biggl(p_i^\gamma + {(1-p_i)}^\gamma\biggr)^{ \frac{1}{\gamma} } }
		\end{equation}
%		where $\gamma > 0$. 
	\item Flexible 2 parameter function \parencite{Prelec1998}
		\begin{equation}
			\label{eq4:pw:pre}
			\omega(p_i)=\exp(-\beta(-\ln(p_i))^\alpha)
		\end{equation}

	\item $\gamma$, $\alpha$ and $\beta$ all $> 0$.

\end{itemize}
\end{frame}


\end{document}
