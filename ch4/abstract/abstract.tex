\documentclass[11pt,a4paper,notitlepage]{report}

\usepackage{setspace}	% For line spacing
\usepackage{csquotes}

\begin{document}

\doublespacing

\begin{abstract}

Shortly after the introduction of Expected Utility Theory, economists and psychologists began publishing results of experiments that showed choices made by experimental subjects which apparently violate one or more of the axioms of Expected Utility Theory.
These responses vary from developing new theoretical models, typically ones that nest Expected Utility Theory as a special case such as Rank Dependent Utility, to critiques of experimental method and scope, to the reemergence of stochastic models of choice.
Over the last several decades, experimental economists have incorporated these insights of theory and practice into their methods in an effort to refine both the hypotheses and evidence associated with these apparent violations.
I conduct a power analysis of the ability of a lottery battery instrument to correctly classify experimental subjects as employing either Expected Utility Theory or Rank Dependent Utility given a popular stochastic choice model, and the effect of this classification on the accuracy of the estimates of welfare surplus for the subjects.
For large ranges of parameter values for these models, I find that the probability of type I and type II errors in the classification process are not trivial, and can be very costly in terms of welfare surplus.
Additionally I show that for a hypothetical population comprising subjects employing Expected Utility Theory or Rank Dependent Utility, we can arrive at more accurate welfare surplus estimates in aggregate by assuming every subject employs the Rank Dependent Utility functional rather than first trying to differentiate Rank Dependent Utility subjects from Expected Utility Theory subjects.

\end{abstract}

\end{document}
