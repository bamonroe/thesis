\documentclass[11pt,a4paper,notitlepage]{report}

\usepackage{setspace}	% For line spacing
\usepackage{csquotes}

\begin{document}

\doublespacing

\begin{abstract}

Economists concerned about potential violations of the independence axiom of Expected Utility Theory (EUT) have proposed experiments to test if subjects are instead more likely to behave as if they weight probabilities, as is proposed by Cumulative Prospect Theory and Rank Dependent Utility (RDU) theory.
Generally these experiments involve subjects making binary choices between two lotteries or gambles, structurally estimating the alternative utility models, and choosing a "winning" model among the alternatives estimated.
I conduct a power analysis of the capacity of a lottery battery instrument to correctly classify experimental subjects as employing either EUT or RDU, and the effect of this classification on the accuracy of subjects' welfare surplus.
For large ranges of parameter values for these models, I find that the probability of type I and type II errors in the classification process are non-trivial, and can be very costly in terms of welfare surplus.
Additionally I show that for a hypothetical population comprising subjects employing EUT or RDU, we can arrive at more accurate welfare surplus estimates in aggregate by assuming that every subject employs the RDU functional, rather than by first trying to differentiate RDU subjects from EUT subjects.

\end{abstract}

\end{document}
